%%%%%%%%%%%%%%%%%%%%%%%%%%%%%%%%%%%%%%%%%%%%%%%%%%%%%%%%%%%%%%%%%%%%
%% Author: Martin Kravec <428724@mail.muni.cz>
%% UČO: 428724
%%%%%%%%%%%%%%%%%%%%%%%%%%%%%%%%%%%%%%%%%%%%%%%%%%%%%%%%%%%%%%%%%%%%

% @TODO: Odsazeni

\documentclass[
	11pt, oneside, printed, final, 
	table,   %% Causes the coloring of tables. Replace with `notable` to restore plain tables.
	lof,     %% Prints the List of Figures. Replace with `nolof` to hide the List of Figures.
	lot     %% Prints the List of Tables. Replace with `nolot` to hide the List of Tables.
	%% More options are listed in the user guide at
	%% <http://mirrors.ctan.org/macros/latex/contrib/fithesis/guide/mu/phil.pdf>.
]{fithesis3}

%%%%%%%%%%%%%%%%%%%%%%%%%%% DEPENDENCIES %%%%%%%%%%%%%%%%%%%%%%%%%%%

\usepackage[czech]{babel}
\usepackage[T1]{fontenc}
\usepackage[utf8]{inputenc}
\usepackage[plainpages=false,pdfpagelabels,unicode]{hyperref}
\usepackage{bold-extra}
\usepackage{setspace}
\usepackage{lastpage}
\usepackage{anyfontsize}
\usepackage{graphicx}

%%%%%%%%%%%%%%%%%%%%%%%%%%% THESIS SETUP %%%%%%%%%%%%%%%%%%%%%%%%%%%

\thesissetup{
	university    = mu,
    faculty       = phil,
    department    = Kabinet informačních studií a knihovnictví,
    field         = Informační studie a knihovnictví,
    type          = bc,
    author        = Martin Kravec,
    gender        = m,
    advisor       = Mgr. Michal Denár,
    date			  = 2016/03/12,
    keywords	  	  = {open source, Koha, Aleph, knihovní system},
	TeXkeywords	  = {open source, Koha, Aleph, library system},
	%keywords	En    = {open source, Koha, Aleph, library system},
	TeXkeywordsEn = {open source, Koha, Aleph, library system},
    title         = Studie proveditelnosti implementace open source integrovaného knihovního systému Koha v~knihovnách Masarykovy univerzity,
    titleEn		  = Feasibility study for the implementation of open source integrated library system Koha in libraries of Masaryk University,
    TeXtitle      = Studie proveditelnosti implementace open source integrovaného knihovního systému Koha v~knihovnách Masarykovy univerzity,
    TeXtitleEn      = Studie proveditelnosti implementace open source integrovaného knihovního systému Koha v~knihovnách Masarykovy univerzity
}

%%%%%%%%%%%%%%%%%%%%%%%%%%%%%% THANKS %%%%%%%%%%%%%%%%%%%%%%%%%%%%%%

\thesislong{thanks}{
    Chtěl bych poděkovat vedoucímu práce Mgr. Michalovi Denárovi za to, že mi ukázal, jak krásně by mohlo fungovat české knihovnictví, a také Mgr. Petře Žabičkové v~roli konzultantky za to, že mi ukázala, jak je na tom české knihovnictví ve skutečnosti. Dále bych rád poděkoval RNDr. Michalovi Růžičkovi, který se chopil role konzultanta a poskytl nedocenitelné informace. A~v~neposlední řadě si zaslouží můj vděk také zaměstnanci univerzitních knihoven MU, kteří jako respondenti ochotně spolupracovali při provádění rozhovorů a stínování v~rámci mého průzkumu mapování knihovních procesů v~knihovnách Masarykovy univerzity.
}

\thesislong{abstract}{ % ANNOTATION!
    Cílem této práce je zejména provést studii proveditelnosti, která určí, zda je možná migrace z~knihovního systému Aleph na open source integrovaný knihovní systém Koha v~knihovnách Masarykovy univerzity. Celá studie je založena na šestiměsíčním průzkumu týkajícího se mapování knihovních procesů ve vybraných knihovnách univerzity, a to formou rozhovorů a stínování. Průzkum byl také prováděn i v~síti 900 knihoven v~Turecku. Studie v~úvodu rozebírá analýzu trhu, na kterou pak navazuje finanční analýza propojená s~řízením a minimalizací rizik. Součástí studie proveditelnosti je také evaluace efektivity a udržitelnosti projektu založena na SWOT analýze. Práce také pojednává o~projektovém managementu a řízení lidských zdrojů a podrobněji rozebírá technické a technologické řešení projektu, přičemž objasňuje aspekty vývoje open source softwaru. 
}

\thesislong{abstractEn}{ % EN ANNOTATION!
    Cílem této práce je zejména provést studii proveditelnosti, která určí, zda je možná migrace z~knihovního systému Aleph na open source integrovaný knihovní systém Koha v~knihovnách Masarykovy univerzity. Celá studie je založena na šestiměsíčním průzkumu týkajícího se mapování knihovních procesů ve vybraných knihovnách univerzity, a to formou rozhovorů a stínování. Průzkum byl také prováděn i v~síti 900 knihoven v~Turecku. Studie v~úvodu rozebírá analýzu trhu, na kterou pak navazuje finanční analýza propojená s~řízením a minimalizací rizik. Součástí studie proveditelnosti je také evaluace efektivity a udržitelnosti projektu založena na SWOT analýze. Práce také pojednává o~projektovém managementu a řízení lidských zdrojů a podrobněji rozebírá technické a technologické řešení projektu, přičemž objasňuje aspekty vývoje open source softwaru. 
}

%%%%%%%%%%%%%%%%%%%%%%%%%%% BIBLIOGRAPHY %%%%%%%%%%%%%%%%%%%%%%%%%%%

\usepackage{csquotes}
\usepackage[
	backend 		= biber,
  	sortlocale	= cs_CZ,
  	style		= iso-authortitle,
  	%style		= iso-numeric,
  	%citestyle=numeric-comp
  	bibencoding = UTF8,
  	babel		= other	% to support multiple languages in bibliography
]{biblatex}
\addbibresource{bibliography.bib}
%% `style`s and `citestyles`, see:
%% <http://mirrors.ctan.org/macros/latex/contrib/biblatex/doc/biblatex.pdf>.

%%%%%%%%%%%%%%%%%%%%%%%%%%% NEW CITATION COMMANDS %%%%%%%%%%%%%%%%%%%%%%%%%%%

\newcommand{\citepages}[2]{[\cite[#1]{#2}]}

\newcommand{\citesource}[1]{[\cite{#1}]}


%%%%%%%%%%%%%%%%%%%%%%%%%%% NEW FORMATING COMMANDS %%%%%%%%%%%%%%%%%%%%%%%%%%%

\newcommand{\bold}[1]{\textbf{#1}}
\newcommand{\italic}[1]{\textit{#1}}
%\newcommand{\citation}[1]{„\italic{#1}"} % @TODO proc nefunguje?
\newcommand{\mezera}{\bigskip}


%%%%%%%%%%%%%%%%%%%%%%%%%%%%%% INDEX %%%%%%%%%%%%%%%%%%%%%%%%%%%%%%

\makeindex

%%%%%%%%%%%%%%%%%%%%%%%%%%%% PROHLASENI %%%%%%%%%%%%%%%%%%%%%%%%%%%

\makeatletter\thesis@load
  \makeatletter\def\thesis@czech@declaration{%
  Prohlašuji, že jsem předkládanou práci zpracoval%
  \thesis@czech@gender@koncovka\ samostatně~a použil%
  \thesis@czech@gender@koncovka\ jen uvedené prameny~a
  literaturu. Současně dávám svolení k~tomu, aby
  elektronická verze této práce byla zpřístupněna přes
  informační systém Masarykovy univerzity.}
\makeatother

%%%%%%%%%%%%%%%%%%%%% DOCUMENT STRUCTURE %%%%%%%%%%%%%%%%%%%%

\makeatletter
  \def\thesis@blocks@preamble{%
    \thesis@blocks@coverMatter
  \thesis@blocks@cover
  \thesis@blocks@titlePage
    \thesis@blocks@frontMatter
      {\Large{Bibliografický záznam}}\newline\newline
        KRAVEC, Martin. \italic{\thesis@title}. Brno: Masarykova univerzita, Filosofická fakulta, 2016. \pageref{LastPage} s. Vedoucí diplomové práce Mgr. Michal Denár.
	  \newline\newline
      {\Large{Anotace}}\newline\newline
        \thesis@abstract\newline\newline
      {\Large{Klíčová slova}}\newline\newline
        \thesis@keywords\newline\newline
      {\Large{Annotation}}\newline\newline
        \thesis@abstractEn\newline\newline
      {\Large{Keywords}}\newline\newline
        \thesis@TeXkeywordsEn\newline\newline
  \thesis@blocks@declaration
  \thesis@blocks@thanks
  \thesis@blocks@clear
	\tableofcontents}
  \def\thesis@blocks@postamble{%
    \thesis@blocks@lot
    \thesis@blocks@lof}
\makeatother

% Je potreba dotahnout tikz.
%\thesis@require{tikz}
%\thesis@require{geometry}
%\geometry{top=15mm,bottom=10mm,left=15mm,right=15mm,includeheadfoot}
  
%%%%%%%%%%%%%%%%%%%%%%%%%%%%% DOCUMENT %%%%%%%%%%%%%%%%%%%%%%%%%%%%%
\begin{document}{

\chapter{Uvod}

Tato bakalářská diplomová práce se bude zabývat studií proveditelnosti implementace open source integrovaného knihovního systému Koha v~knihovnách Masarykovy univerzity. V~práci bude probíráno několik úhlů pohledu na celkovou problematiku, a to v~kontextu studie proveditelnosti.  V~práci tedy rozebereme nejdřív teoretickou část knihovních systémů, co by měly umět a jaký je jejich cíl jako produktu softwarových firem, to znamená, že probereme technické zastřešení projektu a vysvětlíme veškeré pojmy s~tím spojené. Teoretická část dále vysvětlí oblast projektového managmentu a přiblíží problematiku migrací na nový knihovní systém. Celý projekt bude řešen jako harmonogram realizace spolu s~finanční a ekonomickou analýzou. V~rámci studie byl prováděn několikaměsíční průzkum v~knihovnách Masarykovy univerzity a průzkum v~síti 900 knihoven v~Turecku. Při průzkumu byla realizována také SWOT analýza, na které je pak založeno celé hodnocení efektivity a udržitelnosti projektu, ve kterém nebude chybět analýza a řízení případných rizik, které s~sebou projekt migrace nese. V~plánu je také zkrácený výstup v~angličtině ve formě článku, jelikož je to téma s~globálním dopadem, které může pomoci i institucím v~zahraničí, které řeší stejný nebo podobný problém.

\chapter{Teoretická část}

\section{Projektové řízení jako takové}

\subsection{Projekt}

Projektové řízení neboli projektový management se zabývá řízením projektů. Jeho cíle lze rozdělit na 3 části: 
\mezera
\begin{itemize}
\item splnění požadavků
\item časový plán
\item rozpočtové náklady
\end{itemize}

Tomuto rozdělení se také říká trojimperativ\citepages{5}{rosenau_2000}. O~každém projektu lze prohlásit, že je jedinečný, jelikož je prováděn jen jednou, pracují na něm jiní lidé a je časově ohraničen. Projekt je realizován pomocí lidských a materiálních zdrojů. To znamená, že projektový manager vlastně řídí lidi tak, aby byly efektivně využity materiální zdroje\citepages{28}{rehacek_2013}. Následující obrázek 1.1 znázorňuje kvalitu provedení v~kontextu trojimperativu.% @TODO label na obrazek

\begin{figure}[H]
\centering
%egraphics[width=0.95\textwidth,frame]{Resources/Trojimperativ} % @TODO obrazek
\caption{Důsledky trojimperativu.}
\end{figure}

Projekt může být charakterizován jako hmotný či nehmotný projekt v~závislosti na konečném produktu. V~případě hmotného projektu by šlo například o~hardware, u~nehmotného o~software. Produkt je tedy konečným výstupem každého projektu.
V~případě, kdy je zákazník mimo organizaci, která projekt realizuje, jsou projekty realizovány na základě uzavřených smluv, výsledný produkt je tedy posuzován na základě požadavků zákazníka. Je-li zákazníkem stejná organizace, není potřeba uzavírat smlouvy. V~tom případě je ale výstup projektu posuzován podle času dokončení a návratnosti investice\citepages{10-12}{rosenau_2000}.  V~projektovém řízení se zákazník jako objednatel projektu označuje jako zadavatel projektu. Způsob provedení projektu také závisí na konkurenci. Situace, kdy je produkt na trhu jediný svého druhu, je úplně jiná, než u~produktu, kterých je v~oboru mnoho.\citepages{12}{rosenau_2000}
Projektovému řízení se také říká “řízení projektů” nebo “řízení programu” v~závislosti na velikosti projektu. Obecně platí, že programy jsou větší než projekty a projekty větší než úkoly.\citepages{12}{rosenau_2000}
Úspěšnost projektu je závislá na kompetencích managera (nebo leadera) projektu\citepages{1}{Muller2010437}%@TODO pridat strany
, které lze rozdělit na: 
\mezera
\begin{enumerate}
\item Managerské kompetence
\begin{itemize}
\item Efektivní řízení zdrojů
\item Podporování komunikace v~týmu
\item Zmocnění (leader dává svým přímým podřízeným autonomii při řešení problémů, čímž přispívá k~rozvoji jejich \item vlastní zodpovědnosti)
\item Rozvoj (leader nabádá ostatní, aby si brali stále náročnější úkoly a sám investuje svůj čas a úsilí do rozvoje jejich kompetencí)
\item Odhodlání (leader ukazuje své neochvějné odhodlání při dosahování cílů a realizaci projektu)
\end{itemize}
\item Intelektuální kompetence
\begin{itemize}
\item Kritické myšlení (sběr relevantních informací z~různých zdrojů a hledání výhodných a nevýhodných řešení)
\item Vize a představivost (leader má jasnou vizi o~budoucnosti projektu)
\item Strategická perspektiva (leader si je vědom problémů a jejich důsledků, nachází příležitosti a hrozby)
\end{itemize}
\item Emocionální kompetence
\begin{itemize}
\item Sebepoznání (leader je schopen ovládat své pocity)
\item Emocionální odolnost (leader je schopen udržovat stálý výkon ve všech situacích)
\item Intuitivnost
\item Motivace (leader musí svými činy působit motivačně)
\item Svědomitost (leader ukazuje svoji angažovanost k~projektu a sám povzbuzuje ostatní členy pracovní skupiny)
\end{itemize}
\end{enumerate}

\subsection{Proces řízení projektu}
Řízení projektu sestává z~procesu pěti managerských činností, a to\citepages{22-23}{rehacek_2013}:
\mezera
\begin{itemize}
\item Definování projektových cílů
\item Plánování splnění cílů, tedy trojimperativu
\item Efektivní vedení lidských zdrojů (podřízení, dodavatelé)
\item Monitorování - sledování odchylek od plánu či stavů jednotlivých fází projektu
\item Ukončení - finalizace ve smyslu kontroly, zda produkt odpovídá definicím zadavatele, dokončení prací (například dokumentace)
\end{itemize}

Vzájemné závislosti v~procesu řízení projektu znázorňuje obrázek 1.2. % @TODO label na obrazek
Celému cyklu však předchází tzv. předprojektová fáze, do které patří zejména studie proveditelnosti, o~které pojednává další kapitola.

\begin{figure}[H]
\centering
%egraphics[width=0.95\textwidth,frame]{Resources/project-management} % @TODO obrazek
\caption{Vzájemné závislosti v~procesu řízení projektu.}
\end{figure}

\section{Studie proveditelnosti obecně}

Hlavním účelem studie proveditelnosti (feasibility study) je zhodnotit možné varianty, které mohou nastat při provádění projektu, a posoudit realizovatelnost a následnou životaschopnost zvoleného řešení. Studie proveditelnosti zpřesňuje vlastnosti projektu a to hlavně specifikaci cíle, potřebné finanční, materiální a lidské zdroje, časový harmonogram, přínosy a rizika spojená s~realizací projektu. Jelikož studie pojednává o~finančních, technických, managerských i ekonomických aspektech projektu, nazývá se také studií ekonomicko-technickou\citepages{19}{fotr_1995}.
Vyhodnocení studie tedy vyúsťuje do rozhodnutí o~zamítnutí či přijetí projektu a případně jeho následné realizaci.
\citepages{19-20}{fotr_1995} 

\mezera

„\italic{Významné je, aby studie co nejlépe popisovala, variantně řešila, optimalizovala a hodnotila investiční projekt se všemi z~něj vyplývajícími specifiky.}“\citepages{8}{Sieber2004} 

\mezera

Celková osnova studie proveditelnosti by měla vypadat přibližně takto:\citepages{8-14}{Sieber2004} 
\mezera
\begin{enumerate}
\item \bold{Obsah} - struktura kapitol
\item \bold{Úvodní informace} - účel zpracování studie proveditelnosti
\item \bold{Stručné vyhodnocení projektu} - závěry studie v~rozsahu 1-2 stran, zhodnocení finanční efektivity projektu
\item \bold{Stručný popis podstaty projektu a jeho etap} - komplexně pojednává o~hlavních rysech projektu (název, smysl, zaměření projektu, výsledný produkt a problémy, které produkt řeší, lokalizace a etapy projektu)
\item \bold{Analýzy trhu, odhad poptávky, marketingová strategie a marketingový mix} - marketingové aspekty (potřeby finálních uživatelů produktu, konkurenceschopnost produktu)
\item \bold{Management projektu a řízení lidských zdrojů} - plánování, organizace a management procesů a lidských zdrojů projektu 
\item \bold{Technické a technologické řešení projektu} - výhody a nevýhody zvolených technologií, materiálové a energetické toky, technická rizika, náklady na údržbu, správu a provoz.
\item \bold{Dopad projektu na životní prostředí} - kladné i negativní vlivy jednotlivých etap realizace projektu
\item \bold{Zajištění dlouhodobého majetku} - výše investičních nákladů, struktura dlouhodobého majetku
\item \bold{Řízení pracovního kapitálu (oběžný majetek)} - velikost a struktura oběžného majetku
\item \bold{Finanční plán a analýza projektu} - celkové zobecnění předchozích bodů
\item \bold{Hodnocení efektivity a udržitelnosti projektu} - evaluace projektu na základě zadaných kritérií, finanční toky a doba návratnost investic
\item \bold{Řízení rizik (citlivostní analýza)} - výčet zdrojů rizik projektu, opatření
\item \bold{Harmonogram projektu} - časový harmonogram jednotlivých etap projektu. Začátky a konce jednotlivých činností
\item \bold{Podrobné závěrečné hodnocení projektu} - komplexní vyjádření k~realizovatelnosti projektu
\end{enumerate}

Vyhodnocení finanční rentability projektu vybranými ukazateli se provádí tak, že se nejdřív určí jednorázové náklady na investice, odhadnou se budoucí výnosy z~investice, následně se určí náklady na kapitál a nakonec se vypočítá současná hodnota očekávaných výnosů. Na to se pak různě aplikují metody ekonomického hodnocení investice.\citepages{17-18}{Podesvova2010thesis}

\section{Open source jako systém}
Open source systém lze chápat ve dvou rovinách, a to z~pohledu teorie systémů jako systém, který je založen na otevřeném svobodném kódu a slouží pro využívání, zpracování a zprostředkování informací, a také z~pohledu systematičnosti vývoje, kam patří správa verzí, bug trackery, financování, licencování, vytváření balíčků či spolupráce s~komunitou.\citepages{40}{cejpek_2005}

\subsection{Open source software}

Pojem „open source software” bývá častokrát nesprávně zaměňován s~pojmem „free software”, je proto důležité vymezit si rozdíly v~těchto pojmech. 

\bold{Open source software} (OSS) je software, jehož zdrojový kód může být kýmkoliv upravován nebo vylepšován. Open source licence totiž povoluje legální přístup, možnost modifikace či další distribuce tohoto softwaru. Nemalá část komerčního softwaru vychází právě z~open source projektů.

\bold{Free software} je software poskytovaný zdarma, což však neznamená, že má uživatel licenci k~přístupu, modifikaci či k~další distribuci tohoto softwaru. Je definován spíše v~kontextu svobody než ceny. 

Opakem free softwaru je software, jehož zdrojový kód může být modifikován pouze jeho autorem či organizací, která ho vydala. Tomu se říká \bold{closed source software} neboli \bold{proprietární software} (od slova „property”, což znamená, že zdrojový kód softwaru je majetkem svých autorů a jenom oni mohou legálně kopírovat či modifikovat tento software). Příkladem takového softwaru je například Microsoft Office.% @TODO add citation
% McLean, A. (2015). Open-source software. Canadian Journal of Nursing Informatics, 10(3) Retrieved from http://ezproxy.techlib.cz/login?url=http://search.proquest.com/docview/1753599175?accountid=119841

Všeobecně lze OSS licence rozdělit do tří kategorií a to:\citepages{9-10}{6226510}
\begin{itemize}
\item restriktivní neboli \bold{copyleft} licence (omezující)
\item středně restriktivní neboli \bold{copycenter} licence
\item \bold{permisivní} licence (tolerantní, shovívavé)
\end{itemize}
Teoreticky však existuje nekonečné množství těchto licencí, jelikož si autoři open source softwaru mohou vybrat jakoukoliv licenci nebo si vytvořit licenci vlastní.
Na základě výše řečeného lze tvrdit, že open source software je podmnožinou free softwaru, avšak free software nemusí být open source.

\subsection{Financování}
Financování vývoje open source softwaru probíhá častokrát neformálně z~prostředků firem.

\mezera 
„\italic{Firmy si začali uvědomovat výhody, které z~open source softwaru plynou, a stále častěji se přímým způsobem podílejí na jeho vývoji.}”\citepages{129}{Fogel2012}
\mezera

Karl Fogel (autor bestselleru Tvorba open source softwaru z~osvětové edice CZ.NIC a autor svobodného verzovacího systému Subversion) dodává, že organizace, které mají své softwarové potřeby, zbytečně duplikují své snahy, když nakupují software se stejnou funkcionalitou nebo když ho dokonce vytvářejí samy znovu. Když by se připojili k~open source vývoji, náklady na vývoj by se rozdělily mezi všechny další participanty a z~prospěchu by tak čerpali všichni. Tento proces se navíc zdá být prospěšným nejen pro neziskové organizace, ale i pro výdělečné činnosti, kde jako příklady uvádí http://www.openadaptor.org/ a http://koha.org/.\citepages{130-132}{Fogel2012} V~komerční sféře to už ale kvůli licencím možné není.

\subsection{Správa verzí}
Systém pro správu verzí (version control system) je jistým standardem při vývoji open source softwaru. Verzovací systém je nástroj, který primárně spravuje vydávání nových verzí (release management), odděluje stabilní kód od testovacího, respektive produkční kód od vývojového, pomocí takzvaných větví. Mimo jiné pomáhá při opravách chyb a také komunikaci mezi vývojáři.\citepages{17-21, 59-64}{Chaconc2009} Mezi nejznámější verzovací systémy patří Git, Subversion (SVN) a Mercurial.

\subsection{Systém pro sledování chyb}
Spolu se systémy na správu verzí bývají propojené (ale mohou fungovat samostatně) i systémy na sledování chyb, neboli bug trackery\footnote{Pojem bug vystihující skrytou chybu vznikl na Harvardské univerzitě, kdy byl v~jednom z~prvních počítačů nalezen hmyz (štěnice).}. Tento software slouží jako nástroj pro zaznamenávání „problémů”, u~kterých lze definovat počáteční a koncový stav. Slouží nejen pro zaznamenávání chyb, ale také pro požadavky na novou funkcionalitu či vylepšení již stávající. Tyto problémy mohou procházet množstvím předdefinovaných stavů jako například „investigating”, „being worked on”, „near completion” a podobně. Tyto záznamy problému lze různě kategorizovat (například frontend, backend), přidávat jim zadavatele, řešitele, přidávat komentáře a mnoho dalšího. Mezi nejznámější patří Bugzilla, Jira a Buggenie.\citepages{99-104}{Fogel2012}

\section{Migrace na nový knihovní systém}

\subsection{Migrace obecně}

Migrací se rozumí proces, při kterém dochází k~přesunu aplikací integrovaného knihovního systému (dále jen ILS - Integrated Library System) z~jednoho ILS na jiný, který lépe vystihuje potřeby knihovny. V~minulosti se při migraci nahlíželo spíše na vývojové požadavky pro cirkulaci, katalogizaci, akvizici a podobně. Dnes se hodně přihlíží také k~službám, praxi a potřebám a očekáváním uživatele.\citepages{151-152}{bilal_2014}

Důvody k~migraci mohou být různé:\citepages{152}{bilal_2014} % @TODO doplnit jeste M. Bartošek. Aleph: nový knihovní systém pro MU. Zpravodaj ÚVT MU. ISSN 1212-0901, 2002, roč. XIII, č. 2, s. 6-10., http://ics.muni.cz/bulletin/clanky_tisk/263.pdf
\mezera
\begin{itemize}
\item stávající knihovní systém je tradičního charakteru ve smyslu jeho uživatelského rozhraní a funkcí; knihovna se snaží modernizovat, vylepšit uživatelské rozhraní jak po stránce architektury, tak po stránce přívětivosti, chce nabídnout uživatelům discovery služby, které jsou založené na robustních modulech, které lze snadno přizpůsobit
\item dodavatel nebo podpora systému je neuspokojivá
\item závislost na dodavateli (vendor lock-in)
\item systém nebo jeho moduly (neprůhledný finanční model) jsou drahé, návratnost investic je neuspokojivá
\item otázka cloudu vs. lokálního úložiště
\item stávající systém je zastaralý a dodavatel systému již tento produkt nepodporuje
\item knihovna je součástí konsorcia, které rozhodlo o~migraci
\item kapacitní omezení, zpracování velkého množství dat
\item provozní spolehlivost
\item systémová omezení (provoz a bezpečnost)
\item nedostatečná podpora standardů a nových technologií (MARC, RDA, Z39.50, NCIP\footnote{NISO Circulation Interchange Protocol (National Information Standards Organization Circulation Interchange Protocol)
.}, PSGI/Plack\footnote{PSGI/Plack - http://plackperl.org/})
\item omezené možnosti integrace (napojení na národní souborný katalog, národní autority, digitální knihovny)
\end{itemize}

Opakem nedostatečné funkcionality bývá jako další důvod k~migraci také příliš komplikovaný systém či personál, který s~ním reálně neumí zacházet. Hlavním důvodem pro výběr open source systému bývá hlavně cena. Volba konkrétního open source systému je pak ovlivněna jeho funkcionalitou.

\subsection{Proces migrace}
Úspěšnost migrace závisí na mnoha faktorech. Na začátku je nutné pochopit celý migrační proces a rozplánovat ho na několik etap, kde se mezi první etapy řadí čištění dat. Je nutné pochopit datová schémata záznamů ve všech modulech stávajícího systému, aby bylo možné určit, jak by mohly být tato schémata převedena do nového ILS. Je potřeba vědět, v~jakých formátech či stavech se extrahovaná data nacházejí, a také mít spolehlivou komunikaci s~poskytovatelem či podporou systému. Při extrakci dat musí být jasné, jak lze data interpretovat, což znamená napsat extrakční skripty tak, aby skutečně odpovídali nové datové struktuře ILS.\citepages{43-48}{krbiwFHfrnvG2ZXf} Při splnění těchto bodů se přechází k~samotné migraci dat. Tuto migraci je vhodné provádět opakovaně s~různým množstvím dat a importovaná data okamžitě testovat na nějakých tomu určených testovacích stránkách. Po samotném převodu dat je na místě realizace několika testovacích strategií.\cite{32-35, 61-72}{Denar2015thesis} Celkový proces migrace si vyžaduje i svůj “člověkočas”, na který je potřeba být připraven.


Celkový proces migrace vyžaduje svůj čas jak na přípravu, tak na realizaci. V~přípravě nesmí chybět seznam problémů, kterými procházely jiné instituce, které podobnou migrací provedly. Také je vhodné mít případnou záložní „roll back” strategii v~případě neúspěšného pokusu migrace.

\subsection{Obecné požadavky na knihovní systém}
Z~pohledu správce lze definovat jako požadavky na knihovní systém následující.
Systém musí v~rámci modularity poskytovat: \citepages{8-12}{bilal_c2014}

\subsubsection{Modul pro cirkulace}
Tento modul by měl minimálně nabízet funkcionalitu pro kontrolování a vypůjčování knihovního fondu, podporovat pokuty a poplatky, notifikace, rezervace a objednávky, správu čtenářského účtu, správu statistik.

\subsubsection{Akviziční modul}
Tento modul by měl spravovat objednávání jednotek do knihovního fondu. To zahrnuje objednávání, příjem, fakturace, reklamace, alokování fondu či sledování prodejců. 

\subsubsection{Modul pro seriály}
Modul slouží pro správu ročníků, periodik a novin. Může zahrnovat zrušení předplatného, přiřazování opožděných čísel časopisů, alokování fondu, sledování prodejců nebo přečíslování seriálů. V~modulu lze prohledávat fond podle různých parametrů (například ISSN\footnote{International Standard Serial Number}, nebo názvu).

\subsubsection{Modul pro meziknihovní výpůjční službu (MVS)}
Modul pro MVS je určen na půjčování a vypůjčování jednotek mezi knihovnami. Není-li titul dostupný v~rámci knihovny, může knihovna na žádost uživatele zažádat o~meziknihovní výpůjčku. Tyto výpůjčky jsou založeny na smlouvách, které mezi sebou knihovny uzavírají. Ne všechen fond musí být možný vypůjčit.

\subsubsection{Autoritní modul}
Tento modul může být součástí katalogizačního modulu a jeho hlavní funkcionalitou je tvorba a úprava hesel (jména autorů, titulků, seriálů, předmětů ve smyslu témat) pro bibliografické záznamy. Propojuje autoritní heslo s~autoritním záznamem pomocí standardizovaného seznamu hesel.

\subsubsection{Katalogizační modul}
Modul je určen pro zkatalogizování dokumentů a uložení metadat s~následným zařazením do knihovního fondu. Momentálně se využívá formát MARC21, kterého zápis byl donedávna definován pravidly AACR2 (angloamerická katalogizační pravidla verze 2).\citepages{26-27}{dilhofova_kratochvilova_lidmila_2013} Tyto pravidla jsou dnes nahrazena novým standardem RDA (Resource Description and Access), který je založen na konceptu FRBR (Functional Requirements for Bibliographic Record). To je taky důvod, proč jsou některé záznamy zapsané jinak. Je potřeba říct, že RDA ve všeobecnosti nereprezentuje ani pravidla, ani formát, ale spíš strukturovanou množinu myšlenek ve smyslu toho, jak by měl vypadat bibliografický záznam na to, aby byli splněny potřeby uživatele.
% @TODO \citepages{8}{} ocitovat Library Automation,Core Concepts and Practical Systems Analysis
RDA se stávají fakticky závaznými pravidly v~ČR až ve chvíli, kdy chce instituce například dotaci z~VISKu\footnote{Dotační program Ministerstva kultury ČR - Veřejné informační služby knihoven (VISK).}, kdy musí přispívat do Souborného katalogu České republiky, protože musí dodržovat standardy a pravidla pro to daná a podobně.

Celý knihovní systém by měl interně podporovat správu digitálních informačních artefaktů, jako  elektronických knih, elektronických periodik a další, a také mít možnost vytvářet nové multimediální materiály (video, DVD, Blu-Ray a podobně).

\chapter{Praktická část}

\section{Výchozí stav, zdůvodnění realizace projektu a analýza jeho potřebnosti}

\subsection{O~projektu}

Aktuálnost problematiky tkví v~tom, že společnost Ex Libris, která vyvíjí knihovní systém Aleph, přestává tomuto produktu poskytovat podporu a zároveň vyvíjí produkt jiný, jenž však funguje výhradně v~cloudu\footnote{Cloud je jistá metafora pro Internet, která znamená ukládání a zpřístupňování dat a programů na Internet místo na lokálním stroji.}. Jelikož knihovny Masarykovy univerzity používají jako knihovní systém právě Aleph, je potřeba situaci zhodnotit z~různých pohledů.

Mimo fakt, že data již nebudou ukládána lokálně, tj. data již nebudou uložena na strojích patřících provozovateli\footnote{Možné politické a autorsko-právní problémy.}, ale v~cloudu (navíc pravděpodobně mimo jurisdikci EU)\citepages{16-17}{breeding_2012}, to znamená, že knihovny a další instituce využívající knihovních systémů (televize, archivy) budou mít menší možnosti úprav a rozšíření své instance systému. Tyto problémy by mohla vyřešit právě migrace na open source knihovní systém. Jedním z~kandidátů je open source integrovaný knihovní systém Koha, který tato studie proveditelnosti rozebírá. 

Dalším kandidátem, uváží-li se velikost knihoven MU a open source systémy, které prošli několikaletým vývojem, by byl systém Evergreen. Ten je však orientován na anglosaský svět, což může přinášet různé problémy způsobené omezenou podporou pro abecedy s~diakritikou. Jelikož nedisponují balíčky, je problematický i proces aktualizací%\footnote{Installing the Evergreen server - http://evergreen-ils.org/documentation/install/}.
 Systém nebyl vybrán pro studii také pro nedostupnost technické podpory v~ČR a relativně náročnou instalaci %\footnote{Installing the Evergreen server - http://evergreen-ils.org/documentation/install/}
  systému. Systém navíc nepodporuje stahování autoritních záznamů protokolem Z39.50.% @TODO ocitovat Otevřené knihovní systémy v českých knihovnách, http://itlib.cvtisr.sk/buxus/docs/10_Otevrene%20knihovni.pdf
 Navíc má Evergreen desktopového klienta, což činí správu náročnější.
Další důvod podporující migraci na open source je fakt, že knihovny platí nezanedbatelné částky za podporu stávajícího systému, což je oproti open source systémům markantní rozdíl, který lze investovat do smysluplnějších projektů spojených s~cíli knihoven jakožto vzdělávacími, informačními a kulturními institucemi. Investovat do open source projektů je dnes běžnou praxí i v~komerční sféře, jelikož si firmy začali uvědomovat benefity, které díky těmto projektům čerpají od komunity zpátky. \citepages{129-132}{Fogel2012}

Mezi cílové skupiny tohoto projektu migrace ale nepatří jenom systémoví knihovníci, správci a vedení knihovny, ale také zaměstnanci, kteří potřebují splnit/poskytnout informační služby pomocí práce s~klientem knihovního systému, a v~neposlední řadě i uživatelé knihoven, kterým se musí dostat kvalitní služby poskytované danou institucí nehledě na uživatelovi handicapy.

Tato práce tedy pojednává o~problematice v~rámci kontextu každé z~výše zmíněných cílových skupin. Projekt v~této studii proveditelnosti lze do jisté míry využít i v~dalších institucích v~zahraničí, které se zabývají migrací na jiný open source knihovní systém, avšak konkrétní “case study“ je lokalizován na knihovny Masarykovy univerzity, podléhající zákonům platným v~České republice, a to konkrétně knihovnímu zákonu č. 257/2001 Sb. o~knihovnách a podmínkách provozování veřejných knihovnických a informačních služeb, na který navazuje vyhláška č. 88/2002 Sb., zákon č. 106/1999 Sb., o~svobodném přístupu k~informacím, zákon č. 101/2000 Sb., o~ochraně osobních údajů a o~změně některých zákonu a v~neposlední řadě i autorský zákon č. 121/2000 Sb.

\subsection{Etapy studie proveditelnosti}

Celou studii proveditelnosti lze rozdělit na několik etap. Tou první je průzkum, který je nutný k~nalezení informací týkajících se následujících problémů:

\begin{itemize}
\item jaké procesy probíhají v~sledovaných knihovnách
\item jaké procesy probíhají mezi těmito knihovnami
\item jak tyto procesy souvisí s~externími službami
\item jak tyto procesy interagují s~technologickým okolím
\item jak jednotlivé procesy řeší personál knihoven
\item na jaké problémy naráží personál knihoven při konkrétních procesech
\item jak by chtěl personál knihoven zlepšit již stávající procesy
\item jak personál knihoven spravuje technologie v~jednotlivých knihovnách
\item jak jednotlivé knihovny spolupracují s~Ústavem výpočetní techniky a Knihovnicko-informačním centrem univerzity
\item jak je řešena metodologická podpora při věcné a jmenné katalogizaci
\end{itemize}

Druhou etapou studie je zjištění, jak lze získané informace aplikovat v~novém systému, což je vlastně výstupem neboli produktem celé studie. Do další etapy patří rozpracování zjištěného do sekcí studie (viz. kapitola 1.2 Studie proveditelnosti), na což navazuje poslední etapa v~této předprojektové fázi, a to evaluace všech sekcí studie ve formě podrobného závěru projektu. % @TODO pridat label na kapitolu

\section{Management projektu, řízení lidských zdrojů a harmonogram projektu}

\subsection{Řízení lidských zdrojů}

Zásady dobrého managementu projektu jsou již obecně sepsány v~kapitole 1.1 Projektové řízení jako takové. Při řízení lidských zdrojů na tomto projektu je potřeba brát v~úvahu celkový časový harmonogram a mít detailně rozpracovaný plán realizace projektu. Přejde-li se k~samotné realizaci, je potřeba určit skupiny lidí, kteří budou zodpovědní za určitou část realizace.  % @TODO pridat label na kapitolu

První skupinu by tvořil personál, který rozumí metadatům a standardům, minimálně MARC21 a RDA, který by za dohledu metodologické podpory ÚVT MU prováděl čištění dat. Je potřeba se soustředit zejména na umístění dat na správná místa, do správných polí a zapisovat data ve vhodném kódování, které podporuje i abecedy jiných jazyků. Při čištění by měly být odstraněny překlepy a nesprávné formáty stringů, viz standardy. Skupina 1 a skupina 2 by měly spolupracovat, protože množství dat k~pročištění je nemalé, a to tak, že kde to půjde, bude skupina 2 vytvářet scripty pro hromadnou úpravu záznamů na základě pravidel daných skupinou 1.

Druhou skupinu by tvořili lidé, kteří umí psát dotazy na databázi či pracovat s~API stávajícího systému. Tato skupina musí vyčištěná data připravit ve smyslu napsání exportních scriptů. Těmto vyexportovaným datům je nutné  rozumět. Budou-li správně interpretována, bude možné je správně naimportovat do nového systému.
Třetí skupinu lidí musí tvořit systémoví knihovníci a čtvrtou knihovníci na referenčních a výpůjčních pultech. Tito lidé budou systémovým knihovníkům definovat chování, které má systém splňovat, zatímco systémoví knihovníci budou tyto požadavky konfigurovat. Čtvrtá skupina by také měla vytvořit testovací scénáře a snažit se otestovat nový systém, respektive správnost dat, správné interpretování dat, funkčnost poskytovaných funkcí.

\subsection{Management projektu}
Při řízení lidských zdrojů je také potřeba myslet na vývojový tým, tedy skupinu 2, která bude komunikovat s~komunitou a konzultovat případné problémy s~knihovnami, které již procesem prošly. Tato skupina lidí by měla být v~kontaktu s~předchozí skupinou a v~případě potřeb dopisovat chybějící funkcionalitu, či upravovat tu stávající k~potřebám knihovny.

Projektový manager by měl udržovat stálou komunikaci mezi těmito skupinami a zajistit, aby každý chápal, co obnáší proces migrace a proč k~němu vlastně dochází. Musí být vidět, že vedení projektu je přesvědčeno o~přínosu migrace a že si stojí za svým rozhodnutím. 

\subsection{Harmonogram projektu}
Celková realizace projektu vyžaduje svůj čas a je potřeba na to vyčlenit lidské a materiální zdroje. Lidské zdroje jsou omezeny pracovním časem, jelikož musí řešit i běžnou knihovnickou práci, materiální zdroje zase vyžadují přípravu serverů, jak těch testovacích, určených pro nový systém, tak i těch produkčních pro export dat. Je tedy potřeba zajistit dostatečně velké úložiště dat pro nový systém, výkon a zachovat bezpečnost dat (licence, zákony).Celkově lze rozdělit pracovní skupiny dle následující tabulky.

% @TODO Pridat tabulku - Rozdělení pracovních skupin v procesu realizace projektu

\section{Analýza trhu}
Momentálně se Masarykova univerzita skládá z~devíti univerzitních knihoven, které sestávají z~vyše 110 dílčích knihoven. Dohromady spravují cca 2 000 000 knihovních jednotek a obsluhují cca 50 000 uživatelů. MU nemá žádnou ústřední knihovnu.

Při provádění průzkumu v~knihovnách MU vzešlo několik potřeb a požadavků. Tyto požadavky jsou jak hlavní kritéria, které musí nový systém splňovat, tak i potřeby, které je potřeba splnit v~rámci kvalitního designu služeb pro uživatele. K~těmto požadavkům na systém se řadí i obecné požadavky na knihovní systém, definované v~kapitole 1.4.3 Obecné požadavky na knihovní systém. % @TODO pridat label na kapitolu

\subsection{Potřeby a požadavky akvizice}

\begin{itemize}
\item evidovat adresy dodavatelů, IČ, DIČ
\item možnost veřejně spouštět soutěže pro jednotlivé knihovny
\item možnost dohledat, kolik knih bylo koupeno a z~jakých zakázek (každá katedra má jiné zakázky)
\end{itemize}

\subsection{Potřeby a požadavky katalogizace}

\begin{itemize}
\item ILS musí podporovat protokol na výměnu dat OAI-PMH\footnote{The Open Archives Initiative Protocol for Metadata Harvesting.}
\item provázání s~NKP - stahování národních autorit ČR pomocí FTP nebo OAI-PMH
\item podpora ILS v~katalogizaci elektronických zdrojů (momentálně v~Alephu katalogizují jako knihy)
\item možnost hromadných úprav
\item stahování záznamů Z39.50\footnote{Mezinárodní standard definující protokol pro hledání a výměnu dat mezi dvěma stroji.}
\item podpora RDA
\item možnost uzamknout přístup k~editovací části záznamu či celé záznamy pro personál s~omezenými právy
\item ILS musí podporovat práci s~rejstříky
\item napojení na autority, vybrat autoritní heslo
\end{itemize}

\subsection{Potřeby a požadavky cirkulací}

\begin{itemize}
\item možnost zjistit, zda je uživatel student, jestli má registraci nebo jestli je neaktivní student a kde studoval
\item možnost ověření souhlasu uživatele v~INETu
\item možnost udělovat globální nebo lokální blokace (v~Alephu globální a lokální karty)
\item možnost dohledávat diplomové práce v~ISu
\item podpora více výpůjčních protokolů
\item notifikační systém (email, SMS)
\item možnost propojit ILS se selfchecky
\item dohledávat uživatele (propojit Informační systém IS MU, INET MU (ekonomicko-správní informační systém Masarykovy univerzity a databáze/databázi uživatelů v~novém systému)
\end{itemize}

\subsection{Potřeby a požadavky uživatelů}

\begin{itemize}
\item veřejně dostupný katalog (OPAC) pro uživatele knihovny s~možností správy uživatelského účtu a dalších \item \item možných služeb s~nativní podporou discovery služeb nebo využití Centrálního portálu knihoven
\item napojení OPACu na obálky knih přes API 
\item možnost platit pokuty, dobíjet kredit, propojení se systémem SUPO (systém úhrad pohledávek za osobami)
\item možnost COD (Copy on Demand) služeb v~OPACu
\item možnost opravovat překlepy ve vyhledávacích dotazech a nabízet alternativy
\item sjednotit uživatelské rozhraní pro vyhledávání (momentálně Aleph OPAC, Discovery service a někde uvažují o~vývoji vlastního OPAC)
\end{itemize}

\subsection{Systémové požadavky}

\begin{itemize}
\item možnost napojení systému na externí technologie a služby
\item napojení na open source federativní sytém pro propojování identit Shibboleth
\item ekonomický systém (akviziční modul)
\item personální systém
\item robustnost a vysoka provozní spolehlivost systému
\item flexibilita systému, členění systému do složitějších hierarchických struktur
\item otevřenost systému
\item podpora velkého počet knihovních jednotek, uživatelů, poboček, podpoboček
\end{itemize}

Momentálně lze definovat vztahy jednotlivých systému takto:
\mezera
% @TODO pridat schema Alephu - Schéma propojení jednotlivých částí systému Aleph na MU

\subsection{Další požadavky}

\begin{itemize}
\item možnost zjišťovat půjčovanost dokumentu v~rámci jednotlivých knihoven, ne pouze celkově
\item přehledné uživatelské konto ze strany knihovníka
\item mít k~dispozici API pro napojení vlastních či vytvoření nových aplikací
\item mít k~dispozici dokumentaci systému v~českém jazyce
\item zdokumentovaná konfigurace systému
\end{itemize}


\subsection{Současné problémy}
Při provádění průzkumu bylo zjištěno několik nedostatků neboli problémů, se kterými se personál knihoven musel potýkat. Jedná se o~nejaktuálnější zdrojový kód, tedy ke dni psaní této práce to byl Aleph verze 22. V~drtivě většině byl problém nalézt cestu k~cíli v~rámci architektury systému. Byly to většinou problémy uživatelského rozhraní, kdy jim například při provádění některých hromadných akcí vyhodil systém chybu 45krát, nastal-li 45krát problém. Toto okno bylo potřeba 45krát zavřít. Další problém byl zjistit půjčovanost dokumentu v~rámci jedné knihovny.

Následující výčet reprezentuje vlastní aplikace, které jsou založeny na API Alephu nebo využívají SQL dotazy na databáze Alephu. Tyto aplikace vznikly buď chybějící funkčností v~Alephu, nadbytečným množstvím informací poskytnutých systémem (informační přetížení, redundantní informace) nebo nepřesnou interpretací, kterou si knihovny musely ke svým účelům upravit.

\subsubsection{Aplikace KIC}
Aplikace knihovnicko-informačního centra ústavu výpočetní techniky Masarykovy univerzity, které jsou k~dispozici všem knihovnám MU.

Aplikace pro knihovníky:

\begin{itemize}
\item \bold{Souhlas} – Real-time ověření udělení elektronického souhlasu s~provozními řády knihoven MU v~systému INET, kde je to primárně ukládáno, tzn. s~obejitím případných zpoždění/chyb synchronizace při přenosu do knihovního systému
\item \bold{Průkazka} (starší a nová verze) – Aplikace pro registraci externích čtenářů a vydání čipové karty externího čtenáře. Průvodce celým procesem pro knihovníky, protože to integruje interakci s~externím systémem, kde je pro externisty (tj. lidi „z ulice“ bez jiného vztahu k~MU) potřeba založit účet v~univerzitním systému správy identit, aby se byli schopni přihlásit na počítače v~knihovně a do svého konta v~Aleph OPACu.
\item \bold{Signatury} – aplikace konvertující textový seznam čárových kódů knih na PDF s~barevnými štítky se signaturami (stahují se z~databáze k~daným čárovým kódům, barví se dle nějakých pravidel) k~vytištění a nalepení na knihy
\item \bold{Sbírky} – aplikace pro konfiguraci sbírek, tj. kategorií, do kterých se dají jednotky řadit. Generuje se z~konfiguračního souboru Alephu na serveru, pokud knihovník udělá změnu, vygenerují se z~toho na serveru nové konfigurační soubory, kterými se původní konfigurace přepíší a provedou se další akce, aby se aktivovala.
\item \bold{Závazek / seznam výpůjček} – Přehled závazků (případně včetně historie výpůjček) a stavu registrací osoby napříč všemi knihovnami MU, aby se dalo zjistit, jestli nemá čtenář jinde blokace apod.
\item \bold{Report} – Aplikace na e-mailování různých informací. Uvnitř nakonfigurováno cca 20 různých přehledových sestav (např. seznam dlužníků, záznamy bez jednotek, různé typy chyb v~bibliografických záznamech apod.), které si může každý pracovník naklikat s~parametry, které ho zajímají (dlužníci v~konkrétní knihovně s~dluhem větším než 500 Kč apod.) a případně si je nechat i v~pravidelných intervalech s~těmito parametry nechat posílat na svou e-mailovou adresu.
\item \bold{Items} – Konvertor textové seznamu čárových kódů -> Čárový kód, název, sbírka, signatura, rok, poznámka, poznámka OPAC, interní poznámka, materiál
\item \bold{Záznamy} – Pro zadané sysno vygeneruje HTML výpis záznamu a la zobrazení v~knihovním klientu.
\item \bold{Katalogizace} – Přehlad úprav záznamů katalogizátory (pro rozsah datumů)
\item \bold{Půjčování} – Kdo co kdy půjčoval pro rozsah datumů
\item \bold{Seznam} – Tisk přírůstkového / úbytkového seznamu pro vybranou knihovnu a rozsah čárových kódů nebo datumů
\item \bold{Evidence} – Už se prakticky nepoužívá, vnitřní evidence půjčování několika interně půjčeních knih v~rámci ÚVT
\item \bold{Eprace} – Již nepoužívaná jednoúčelová aplikace. Dříve se používalo pro evidenci zpracování diplomových prací, když se digitalizovali pro elektronickou evidenci. Teď už se přijímají přímo elektronicky.
\item \bold{Toplist} – Seznam nejčastěji půjčovaných knih s~možností volby knihovny, rozsahu datumů, počtu příček a volitelným filtrem dle sbírky, do které jednotky patří
\item \bold{Dlužníci} – Report o~dlužnících (s~možností filtrace dle knihovny, datumů a minimální dlužné částky) ve formátu přímo vložitelném do ekonomického systému Magion
\item \bold{Výpůjčky} – Čtenáři s~nevrácenými výpůjčkami v~prodlení s~možností filtrace dle knihovny a druhů čtenářů (studenti, doktorandi, zaměstnanci apod.)
\item \bold{Selfcheck} – report o~využívání selfchecků s~filtrem dle knihovny a rozsahů datumů
\item \bold{E-prezenčka} – Evidence zpracování pro ePrezenčku
\item \bold{Smazání} – Seznam smazaných čtenářských účtů s~datem vymazání
\item \bold{FAQ} – Různé návody, pokyny ke katalogizaci apod. pro knihovníky (Pamatuje si, co kdo už viděl, takže nepřečtené/aktualizované věci jsou při první návštěvě knihovníkovi zvýrazněny.)
\end{itemize}

Nástroje pro konvertování záznamů:

\begin{itemize}
\item používají hlavně katalogizátoři. Jedná se o~vyhledávací formuláře nad různými MARC dumpy\footnote{Výpisy, exporty.} bibliografických záznamů (často i různých období apod.) s~možností vyhledávání dle různých kritérií (včetně možnosti regulárních výrazů apod.). Vyhledává se Perlem na textovým dumpem MARC záznamů, takže se tam dají udělat různá poměrně komplexní vyhledávání dle složitých podmínek. 
\item aplikace pro kontrolu bibliografických záznamů před odesláním do Souborného katalogu v~Národní knihovně
\end{itemize}

Systémové nástroje:

\begin{itemize}
\item náhled dat z~databáze
\item odkazy na stažení instalační soubory knihovního klienta (aby je měl správce k~dispozici v~aktuální verzi, když se v~knihovně reinstaluje nějaký počítač) a TeamView klienta, když je někomu potřeba vzdáleně pomoci přímo na jeho počítači
\end{itemize}

\subsubsection{Aplikace Fakulty sociálních studií}

\begin{itemize}
\item aplikace pro soutěž akvizicí
\end{itemize}

\subsubsection{Aplikace KUK (knihovny univerzitního kampusu)}
\begin{itemize}
\item \bold{MVS} – správa požadavků
\item \bold{Dlouhodobé výpůjčky} – v~OPACu může uživatel požádat o~dočasnou výpůjčku knihy, kterou má aktuálně půjčenou vyučující dlouhodobě.
\item \bold{Dlouhodobé, které letos končí} – aby bylo možné upozornit včas vyučující, prodloužit výpůjčku apod
\item \bold{Hromadná objednávka} – vytisknout více titulů do jednoho objednávkového formuláře v~designu MU
\item \bold{Rozpočty} – přehled rozpočtů aktuálního roku, stav čerpání financí a zobrazení navázaných titulů/jednotek.
\item \bold{Limity nákupu} – jeden z~rozpočtů má určeny limity pro jednotlivá pracoviště (sbírky) – každé pracoviště má zobrazen stav čerpání svého limitu a nakoupené jednotky.
\item \bold{Poptávky} – velký systém pro sdružené poptávání sad titulů, sběr cenových nabídek a následný tisk objednávek po ukončení soutěže.
\item \bold{Dílčí knihovny} – MU má cca 90 dílčích knihoven (kateder). Zobrazení aktuálních přírůstků, celkového stavu fondu, evidence údajů o~jednotlivých sbírkách pro zobrazení na webu knihovny
\item \bold{Dopis hříšníkům} – na základě seznamu UČO vygenerovat jednotlivé dopisy v~komunikačních jazycích uživatelů (cz/en). Z~Alephu se přebírá seznam dlužených knih a rozpis aktuální dlužené částky Kč
\item \bold{Dary} – musí evidovat osobní údaje lidí, kteří darují knihovně knihu.
\item \bold{Kontrola chyb} – skript se sadou sql dotazů zobrazí očividné chyby – např jednotka bez čárového kódu, se špatně uvedenou cenou, špatným datem nabytí apod.
\item \bold{Kontrola řady čárových kódů} – při zpracování knih se občas ztratí čárový kód, je potřeba ho dohledat a obsadit (jeho číslo je zároveň přírůstkové číslo)
\item \bold{Žebříčky/rezervace} - přehled počtu rezervovaných/vypůjčených dokumentů v~daném roce podle sbírky jednotky, fakultní příslušnosti čtenáře
\item \bold{Přehled fondů pro potřeby akreditací} - vygenerování přehledu počtu svazků knih/časopisů podle majetku fakulty a kolik z~toho je ve volných výběrech
\item \bold{Výpůjční statistiky} - generování přehledu počtu vypůjčených/vrácených dokumentů po hodinách v~jednotlivých dnech + počet čtenářů, kteří v~daný den si něco půjčili/vrátili
\item \bold{Přehled knih pro zahraniční studenty}
\end{itemize}

\subsection{Technické a technologické řešení projektu}

Veškeré požadavky vycházejí z~šestiměsíčního zkoumání fakultních knihoven MU, které bylo prováděno jako moderované rozhovory s~jednou až třemi osobami zároveň a stínování každé z~nich. Rozhovorů se účastnili nejen zaměstnanci, ale i vedoucí knihoven. 

Každý okruh knihovních procesů byl mapován samostatně, nejdříve se zkoumaly služby u~referenčních pultů, pak akvizice, katalogizace a automatizace. Následně byly zkoumány zahraniční knihovny a instituce, které migrací prošli formou rešerší a osobní či e-mailové komunikace v~rámci komunitního maillistu\footnote{E-mailový chat komunity.
.}. 

Tak jako každý knihovní systém, i Koha splňuje základní funkcionalitu jednotlivých modulů definovaných v~kapitole 1.4.3 Obecné požadavky na knihovní systém.% @TODO label na kapitolu

Při prováděném výzkumu bylo zjištěno několik problémů a menších či větších nedostatků. Tyto nedostatky se snaží tato studie vyřešit open source řešeními. Studie počítá se systémem Koha ze stejných důvodů jako jiné knihovny (s~různými komerčními knihovními systémy), které už migrací prošly. Mezi hlavní důvody lze zařadit: 

\begin{itemize}
\item výrazné jednodušší instalace a aktualizace systému oproti open source systému Evergreen
\item instalační balíčky pro Debian a Ubuntu
\item možnost využívat autoritní záznamy při katalogizaci
\item stabilní vývoj, pravidelný cyklus verzí
\item aktivita komunity
\item jasná budoucí vize projektu
\item podpora autorit
\item napojení na discovery VuFind a elektronické informační zdroje (například EBSCO)
\item výborné schopnosti akvizičního modulu
\item přehledný katalogizační editor s~možností rozšíření
\item rozsáhlé možnosti notifikací stavů čtenářům e-mailem a SMS
\item pestré možnosti API včetně jeho rozšiřování
\end{itemize}

Mezi zajímavá zjištění lze zařadit, že drtivá většina knihoven některé moduly, jako je třeba akviziční modul, nepoužívají a nahrazují jej buďto vlastními aplikacemi nebo výše zmíněnými alternativami.

Modul cirkulací s~historií výpůjček byl v~jednom případě vyměněn za fyzické lístečky s~historií výpůjčky dané jednotky umístěnou v~deskách jednotky. 

Potřeby a požadavky MU zpracované v~této kapitole je možné rozpracovat detailněji, což však přesahuje rozsah bakalářské práce a může tak být tématem pro další navazující práci. Tento výčet požadavků a jejich řešení však dostatečně reflektuje potřebné informace k~tomu, aby bylo možné sestavit plnohodnotný závěr studie proveditelnosti.

\subsection{Potřeby a požadavky cirkulací}

Modul cirkulací je v~Koha velmi přívětivý a flexibilní. Uživatele nebo výpůjčku lze dohledat načtením čárového kódu nebo zadáním názvu či kódu jednotky, v~případě uživatele zadáním jména. Výstupem této akce jsou nejdůležitější údaje o~hledaném, například datum vypůjčení dané jednotky, datum pro vrácení jednotky, možnost prodloužit jednu nebo více výpůjček, místo půjčení, typ jednotky a podobně. U~vracení stačí načíst čárové kódy, promíjet upomínky nebo vracet knihy z~biblioboxů.

% @TODO add image - Výpůjčky čtenáře v systému Koha

Koha umí uživatele předem notifikovat pomocí emailu nebo SMS, kde správa šablon různých zpráv je samozřejmostí. Těmto notifikacím lze různě nastavovat například periodicitu a další vlastnosti.

\subsubsection{Napojení na selfchecky}
Koha podporuje protokol SIP2, díky kterému je možné napojit systémy selfchecků, RFID technologie, kopírky/tiskárny či Zebra tiskárny pro tisk štítků.% #TODO add citation Choosing free library software: Experiences of the Faculty of Humanities and Social Studies in Zagreb, s. 651

\subsubsection{Podpora více výpůjčních protokolů}
Jelikož má Koha možnost vytvářet pobočky, má i podporu více výpůjčních protokolů. V~systému Koha je možné nastavovat vazby mezi jednotlivými knihovnami/pobočkami, tudíž je možné nastavit, kdo má na co práva a kam “vidí”. Je tedy možné pobočky propojit nebo naopak rozdělit.

\subsubsection{Lokální a globální karty Alephu}
Tato funkcionalita vyžaduje hlubší šetření. Momentálně se databáze uživatelů sdílí mezi všemi v~rámci instance a jde blokovat jen globálně a to buď na základě poplatků nebo nevrácení. V~případě, kdy by v~rámci instance neměla knihovna vidět do uživatelské databáze jiné knihovny, je potřeba chování systému upravit. Bylo by tedy potřeba vytvořit testovací instanci, která by představovala novou strukturu systému a na ní odladit potřebné požadavky.

\subsubsection{MVS (Meziknihovní výpůjční služba)}
Meziknihovní výpůjční služba funguje v~Koha jako běžná výpůjčka, ale pro speciální typ uživatele - knihovnu. Knihovny jsou speciální kategorie čtenářů s~odpovídajícími parametry, jako cena za nula a pravidly pro výpůjčky. Takže se provede klasická výpůjčka, která se následně odešle poštou. Lze vygenerovat report, který přehledně zobrazí, jaká knihovna od kdy má daný dokument. Příchozí MVS je řešena specifickým typem dokumentu, který má nastaven kratší dobu výpůjčky, navěšeno hlášení při návratu a specifickou katalogizační šablonu. Takže dokument je přijat, rychle zkatalogizován, do jednoho pole je zapsáno, od koho je, a půjčen čtenáři. Opět je k~dispozici report, který vypisuje, jaké dokumenty v~rámci příchozích MVS jsou k~dispozici, kdo je má, atd. Toto fungování nemusí zcela vyhovovat MU a celou agentu bude pravděpodobně potřeba upravit nebo použít MVS modul\footnote{ILL modul Koha - https://github.com/PTFS-Europe/koha/tree/ill\_master}.

\subsection{Potřeby a požadavky uživatelů}
Celý problém lze rozdělit na dvě části, vnitřní a vnější. Tou vnitřní částí je rozhraní systému, se kterým měli potíže zaměstnanci. Tento problém lze v~systému Koha vyřešit velmi elegantně a prakticky bez bariér. Jelikož je rozhraní systému webové, lze ho upravovat pomocí technologií HTML5 a CSS3 spolu s~JavaScriptovou knihovnou jQuery, která dodá celému rozhraní uživatelskou přívětivost (efekty místo statického loadingu, asynchronní načítání obsahu, hezké zobrazování notifikací, slideUp efekty místo klasického smazání části obsahu a mnoho dalšího). Právě JavaScript je jednoduché řešení, po kterém knihovny sahají, potřebují-li si přizpůsobit nějaké řešení svým potřebám.\citepages{80-82}{Denar2015thesis} Do každé stránky rozhraní lze vkládat libovolný javascript, takže upravit lze prakticky všechno. Díky jQuery lze tedy definovat  vlastní události a přiřazovat k~nim nové či pozměněné akce nebo měnit či přidávat efekty.\citepages{3-9}{IYiYMQHMhgWYsu4q}

Vnější část problému je uživatelské rozhraní, tedy OPAC\footnote{Online Public Access Catalog
}, se kterým uživatelé neumí pořádně pracovat. Při provádění výzkumu měl autor studie možnost vyzkoušet si práci referenčního knihovníka na Filozofické fakultě Masarykovy univerzity, z~čehož vyplynulo, že uživatelé OPACu opakovaně nebyli schopni dohledat informace, které potřebovali, a žádali o~pomoc referenční knihovníky. Většinou to byly problémy jako nedohledatelná kniha na regálu, nemožnost inteligentně dohledat knihu na nějaké téma v~ruském jazyce a podobně.

Problém s~jazyky byl ten, že OPAC nabízel například 3 jazyky z~určitého MARC pole, zatímco se reálně prohledávalo pole jiné. V~době psaní této práce existovala někde v~rozhraní tabulka se zkratkami jazyka, které bylo možné použít pro prohledávání. Uživatelé, kteří o~tabulce nevěděli, byli nuceni zvolit jeden z~jazyků, spustit vyhledávání a v~URL změnit zkratku jazyku za jinou, kterou si tipli, že je správná pro daný jazyk.

Tyto problémy lze v~systému Koha vyřešit zaprvé správným zaindexováním dat a zadruhé uživatelsky přívětivým OPACem, kterým se osvědčil být VuFind vyvíjený jako open source na Villanova University.\citepages{12}{Coufalova2009} Tento OPAC používá drtivá většina knihoven v~České republice provozující systém Koha (zbytek knihoven je maximálně nenáročných a zůstali na interním Koha OPACu). VuFind umožňuje uživatelům vyhledávat v~celém fondu knihovny a nabízí i doplňkové služby a to celé přehledně a jednoduše. VuFind je modulární, o~čemž svědčí i fakt, že je na něm postaven český Centrální portál knihoven\footnote{Centrální portál Knihoven (CPK) - https://knihovny.cz/}, který poskytuje přístup k~fondům a službám všech zapojených knihovních i neknihovních institucí v~rámci republiky. Centrální portál knihoven je vyvíjen jako open source a jeho moduly jsou dostupné v~repozitáři na Githubu\footnote{CPK repozitář https://github.com/moravianlibrary/VuFind-2.x
}. Lze se tak inspirovat řešením, které prošlo uživatelským testováním a je založeno na uživatelsky přívětivém moderním designu. % @TODO pridat 2 citace, Aktivity českého knihovnictví  http://koncepce.knihovna.cz/aktivity/ a Smlouva o dílo, https://www.mlp.cz/cz/o-knihovne/zverejnene-smlouvy/?knihovna=&knihovna=0, Prosinec 2015

VuFind je navíc discovery systém, který lze v~rámci Masarykovy univerzity využít k~zapojení externích elektronických informačních zdrojů, což OPAC Kohy neumožňuje. VuFind je také stavěný na práci se Solrem, což je databázový systém vhodný pro práci s~Big Data\footnote{Víc informací o~BigData k~nalezení v~- BigData, VIktorMayer citace
}. Podle Alfreda Serafiniho\footnote{Freelance software konzultant, autor publikace Apache Solr Beginner's Guide (ISBN 978-1-78216-252-0).} je to vhodný nástroj pro prohledávání velkého množství read-only metadat, kde jako příklady využití uvádí právě VuFind a Internet Archive\footnote{Internet Archive - http://www.archive.org/}. % @TODO pridat citaci na Serafini, A 2013, Apache Solr Beginner's Guide, Birmingham: Packt Publishing, eBook Collection (EBSCOhost), EBSCOhost, viewed 31 March 2016. s.20, http://search.ebscohost.com/login.aspx?direct=true&db=nlebk&AN=680715&lang=cs&site=ehost-live&ebv=EB&ppid=pp_20

% @TODO nahore opravit footnote pro BigData

Idea rozhraní VuFind je založena na teorii, že uživatele nezajímá cesta, jakou se k~dokumentu dostane, ale v~jaké formě je k~dispozici a jak je možné ho získat v~rámci instituce.\citepages{20}{Coufalova2009} Lze tedy tvrdit, že katalog, který slouží knihovníkům, nemůže být stejný jako OPAC\citepages{5}{Schmidt2012}, jelikož ten má být designován pro uživatele, kteří s~ním interagují. OPAC je veřejnou částí knihovního systému, tedy vstupní bránou uživatele, která musí být uživatelsky přívětivá, musí tedy splnit potřeby uživatele, které lze vydefinovat pomoci Maslowovy pyramidy webdesignu\citepages{156-179}{rezac_2014}, kam patří důvěryhodnost, radost z~používání, vytvoření vazby, přesvědčivost, použitelnost, přístupnost, dostupnost, nalezitelnost a smysluplnost. 

Tak jako Koha, i VuFind je postaven na webových technologiích, což dává designerům a softwarovým architektům obrovské možnosti v~rámci přizpůsobení veřejného rozhraní. Pomocí HTML5 lze rozhraní zpřístupnit i zrakově či jinak handicapovaným uživatelům, kterým se tak OPAC zpřístupní pro jejich speciální čtecí zařízení. \citepages{19-27}{Hogan2011}

Zajímavostí je také budoucnost těchto technologíí, která vnáší do oboru knihovnictví sémantičnost v~rámci webového rozhraní. Díky HTML5 a sémantickým technologiím, jako je například mikroformát RDF (a také samotná podstata sémantiky v~HTML5), je možné dávat jednotlivým částem webu význam, což z~něho dělá lépe prohledávatelný web na Internetu, na což vzápětí navazuje i chytré vyhledávání na webu za využití umělé inteligence.\citepages{70-77}{fay_sauers_2012} Takto lze OPAC připravit na budoucnost webu 3.0, respektive webu 4.0.

VuFind je navíc založen na myšlence responsivního designu a využívá webového frameworku  Bootstrap 3, je tedy responzivní ve smyslu přístupnosti na zařízeních s~různou uhlopříčkou a rozlíšením, takže je stejně použitelný jak na desktopech, tak i na tabletech či smartphonech.

\subsubsection{Služba Copy on Demand (COD)}
Tato služba přímo v~Koha ani ve VuFindu neexistuje, avšak je natolik triviální, že její napsání nezabere víc jak pár hodin. Tuto funkci lze také řešit externí službou, která vygeneruje odkaz (tlačítko), který stačí jednoduše embedovat\footnote{Vložit.} do šablony OPACu.


\subsubsection{Obálky knih}
Nejen obálky knih, ale i naskenované obsahy a indexy lze zpřístupnit ve VuFindu pomocí API ObálekKnih\footnote{API ObálkyKnih.cz - https://obalkyknih.cz/doc/Dokumentace\_API\_OKCZ\_3.1.pdf.} ve verzi 3. Příklad implementace ve VuFindu lze vidět na Centrálním portálu knihoven, respektive v~repozitáři\footnote{Implementace ObálekKnih v~CPK https://github.com/moravianlibrary/VuFind-2.x/tree/master/themes/obalkyknih-api-v3-bootstrap3} na Gthubu.

\subsubsection{Automatická oprava překlepů a nabídka alternativ při neúspěšném hledání}
Tuto funkcionalitu lze zapnout vyhledávací komponentou „spellcheck”.

\subsection{Systémové požadavky}
Následující část rozebírá hlavní systémové požadavky na knihovní systém Masarykovy univerzity.

\subsubsection{Velký počet knihovních jednotek, uživatelů, poboček, podpoboček}


%%%%%%%%%%%%%%%%%%%%%%%%%%%%% TABLE OF CONTENTS %%%%%%%%%%%%%%%%%%%%%%%%%%%%%

\makeatletter\thesis@blocks@clear\makeatother
\phantomsection %% Print the index and insert it into the
\addcontentsline{toc}{chapter}{\indexname} %% table of contents.

%%%%%%%%%%%%%%%%%%%%%%%%%%%% APPENDIX HYPERREFS %%%%%%%%%%%%%%%%%%%%%%%%%%%%

\makeatletter\thesis@blocks@clear\makeatother
%% Patch the appendix hyperrefs (see:
%% <http://tex.stackexchange.com/q/174887/70941>)
\renewcommand{\theHchapter}{A\arabic{chapter}
\appendix %% and start the appendices.

%%%%%%%%%%%%%%%%%%%%%%%%%%%%%%% BIBLIOGRAPHY %%%%%%%%%%%%%%%%%%%%%%%%%%%%%%%

\printbibliography[title={Seznam literatury},heading=bibintoc]

%%%%%%%%%%%%%%%%%%%%%%%%%%%%%% END OP CONTENT %%%%%%%%%%%%%%%%%%%%%%%%%%%%%%

}
\end{document}
