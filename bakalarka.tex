%%%%%%%%%%%%%%%%%%%%%%%%%%%%%%%%%%%%%%%%%%%%%%%%%%%%%%%%%%%%%%%%%%%%
%% Author: Martin Kravec <428724@mail.muni.cz>
%% UČO: 428724
%%%%%%%%%%%%%%%%%%%%%%%%%%%%%%%%%%%%%%%%%%%%%%%%%%%%%%%%%%%%%%%%%%%%

\documentclass[
	12pt, oneside, printed, final, 
	table,   %% Causes the coloring of tables. Replace with `notable` to restore plain tables.
	lof,     %% Prints the List of Figures. Replace with `nolof` to hide the List of Figures.
	lot     %% Prints the List of Tables. Replace with `nolot` to hide the List of Tables.
	%% More options are listed in the user guide at
	%% <http://mirrors.ctan.org/macros/latex/contrib/fithesis/guide/mu/phil.pdf>.
]{fithesis3}

%%%%%%%%%%%%%%%%%%%%%%%%%%% DEPENDENCIES %%%%%%%%%%%%%%%%%%%%%%%%%%%

\usepackage[czech]{babel}
\usepackage[T1]{fontenc}
\usepackage[utf8]{inputenc}
\usepackage[plainpages=false,pdfpagelabels,unicode]{hyperref}
\usepackage{bold-extra}
\usepackage{setspace}
\usepackage{lastpage}
\usepackage{anyfontsize}

%%%%%%%%%%%%%%%%%%%%%%%%%%% THESIS SETUP %%%%%%%%%%%%%%%%%%%%%%%%%%%

\thesissetup{
	university    = mu,
    faculty       = phil,
    department    = Kabinet informačních studií a knihovnictví,
    field         = Informační studie a knihovnictví,
    type          = bc,
    author        = Martin Kravec,
    gender        = m,
    advisor       = Mgr. Michal Denár,
    date			  = 2016/03/12,
    keywords	  	  = {open source, Koha, Aleph, knihovní system},
	TeXkeywords	  = {open source, Koha, Aleph, library system},
	%keywords	En    = {open source, Koha, Aleph, library system},
	TeXkeywordsEn = {open source, Koha, Aleph, library system},
    title         = Studie proveditelnosti implementace open source integrovaného knihovního systému Koha v~knihovnách Masarykovy univerzity,
    titleEn		  = Feasibility study for the implementation of open source integrated library system Koha in libraries of Masaryk University,
    TeXtitle      = Studie proveditelnosti implementace open source integrovaného knihovního systému Koha v~knihovnách Masarykovy univerzity,
    TeXtitleEn      = Studie proveditelnosti implementace open source integrovaného knihovního systému Koha v~knihovnách Masarykovy univerzity
}

%%%%%%%%%%%%%%%%%%%%%%%%%%%%%% THANKS %%%%%%%%%%%%%%%%%%%%%%%%%%%%%%

\thesislong{thanks}{
    Chtěl bych poděkovat Michalovi Denárovi za to, že mi ukázal, jak krásně by mohlo fungovat české knihovnictví, a také Petře Žabičkové za to, že mi ukázala, jak je na tom české knihovnictví ve skutečnosti.
}

\thesislong{abstract}{ % ANNOTATION!
    Cílem této práce je zejména provést studii proveditelnosti, která určí, zda je možná migrace
z~knihovního systému Aleph na open source integrovaný knihovní systém Koha. Součástí studie
proveditelnosti bude jak finanční plán, tak i hodnocení efektivity a udržitelnosti projektu,
definice možných rizik a navržené způsoby jejich řízení a minimalizace.
}

\thesislong{abstractEn}{ % EN ANNOTATION!
    Cílem této práce je zejména provést studii proveditelnosti, která určí, zda je možná migrace
z~knihovního systému Aleph na open source integrovaný knihovní systém Koha. Součástí studie
proveditelnosti bude jak finanční plán, tak i hodnocení efektivity a udržitelnosti projektu,
definice možných rizik a navržené způsoby jejich řízení a minimalizace.
}

%%%%%%%%%%%%%%%%%%%%%%%%%%% BIBLIOGRAPHY %%%%%%%%%%%%%%%%%%%%%%%%%%%

\usepackage{csquotes}
\usepackage[
	backend 		= biber,
  	sortlocale	= cs_CZ,
  	style		= iso-authortitle,
  	bibencoding = UTF8,
  	babel		= other	% to support multiple languages in bibliography
]{biblatex}
\addbibresource{bibliography.bib}
%% `style`s and `citestyles`, see:
%% <http://mirrors.ctan.org/macros/latex/contrib/biblatex/doc/biblatex.pdf>.

%%%%%%%%%%%%%%%%%%%%%%%%%%% NEW CITATION COMMANDS %%%%%%%%%%%%%%%%%%%%%%%%%%%

\newcommand{\citepages}[2]{[\cite[#1]{#2}]}

\newcommand{\citesource}[1]{[\cite{#1}]}

%%%%%%%%%%%%%%%%%%%%%%%%%%%%%% INDEX %%%%%%%%%%%%%%%%%%%%%%%%%%%%%%

\makeindex

%%%%%%%%%%%%%%%%%%%%%%%%%%%% PROHLASENI %%%%%%%%%%%%%%%%%%%%%%%%%%%

\makeatletter\thesis@load
  \makeatletter\def\thesis@czech@declaration{%
  Prohlašuji, že jsem předkládanou práci zpracoval%
  \thesis@czech@gender@koncovka\ samostatně~a použil%
  \thesis@czech@gender@koncovka\ jen uvedené prameny~a
  literaturu. Současně dávám svolení k~tomu, aby
  elektronická verze této práce byla zpřístupněna přes
  informační systém Masarykovy univerzity.}
\makeatother

%%%%%%%%%%%%%%%%%%%%% DOCUMENT STRUCTURE %%%%%%%%%%%%%%%%%%%%

\makeatletter
  \def\thesis@blocks@preamble{%
    \thesis@blocks@coverMatter
  \thesis@blocks@cover
  \thesis@blocks@titlePage
    \thesis@blocks@frontMatter
      {\Large{Bibliografický záznam}}\newline\newline
        KRAVEC, Martin. \textit{\thesis@title}. Brno: Masarykova univerzita, Filosofická fakulta, 2016. \pageref{LastPage} s. Vedoucí diplomové práce Mgr. Michal Denár.
	  \newline\newline
      {\Large{Anotace}}\newline\newline
        \thesis@abstract\newline\newline
      {\Large{Klíčová slova}}\newline\newline
        \thesis@keywords\newline\newline
      {\Large{Annotation}}\newline\newline
        \thesis@abstractEn\newline\newline
      {\Large{Keywords}}\newline\newline
        \thesis@TeXkeywordsEn\newline\newline
  \thesis@blocks@declaration
  \thesis@blocks@thanks
  \thesis@blocks@clear
	\tableofcontents}
  \def\thesis@blocks@postamble{%
    \thesis@blocks@lot
    \thesis@blocks@lof}
\makeatother
  
%%%%%%%%%%%%%%%%%%%%%%%%%%%%% DOCUMENT %%%%%%%%%%%%%%%%%%%%%%%%%%%%%
\begin{document}{\fontsize{12}{18}

\chapter{Uvod}

Bakalářská diplomová práce se bude zabývat studií proveditelnosti implementace 
open source integrovaného knihovního systému Koha v~knihovnách Masarykovy univerzity. V~praci bude probirano několik udhlu pohledu na celkovou problematiku a to v~kontextu studie proveditelnosti. [Studie proveditelmnosti]. V~praci tedy rozebereme nejdriv teoretickou cast knihovnich systemu, co by měli umet a jaky je jejich cil jako produktu softwarovych firem, to znamena probereme technicke zastreseni projektu. Cely projekt bude resen i jako harmonogram realizace spolu s~financni a ekonomickou analyzou. V~ramci studie byl provaden nekolikamesicni vyzkum ve vybranych knihovnach masarykovy university, na kterem je zalozeno cele hodnoceni efektivity a udrzitelnosti projektu, v~kterem nebude chybet analuza a rizeni pripadnych rizik, které sebou projekt  migrace nese. Cely vystup je také dostupny v~anglictine, jelikoz je to tema s~globalnim dopadem, které muze pomoct i jinych krajinach, které resi stejny, nebo podobny poblem.

\chapter{Teoreticka cast }

\section{Výchozí stav, zdůvodnění realizace projektu a analýza jeho potřebnosti}

Obsahuje stručný popis stávající situace (problémy a nedostatky), kterou má projekt
řešit, poptávku po realizaci projekt/analýzu a definici přínosu/potřebnosti projektu,
jmenuje cílové skupiny, na které bude mít projekt vliv

Aktuálnost problematiky tkví v~tom, že společnost Ex Libris, která vyvíjí knihovní 
systém Aleph, přestává tomuto produktu poskytovat podporu a zároveň vytváří produkt jiný, 
jenž však funguje výhradně v~cloudu. 
Mimo fakt, že data již nebudou ukládána lokálně, to znamená, že knihovny budou mít 
menší možnosti úprav a rozšíření své instance systému. Tyto problémy by pak měla 
vyřešit právě migrace na knihovní systém Koha. 
Další důvod podporující migraci je fakt, že knihovny platí milionové částky za 
podporu stávajícího systému, což je oproti open source systémům markantní rozdíl, 
který lze investovat do smysluplnějších  věcí spojených s~cíli knihoven, 
jakožto vzdělávacími, informačními a kulturními institucemi. 
Mezi cilove skupiny tohoto projektu migrace ale nepatri jenom systemovy knihovnici, spravci a vedeni knihovny ale také zamestnanci který potrebuji splnit / poskytnout informacni sluzby pomoci práce s~klientem knihovniho systemu, a v~neposlednim pripade i uzivatele knihoven, kterym se musí dostat kvalitni sluzby, poskytovane danou instituci. 
Tato práce tedy pojednava o~problematice v~ramci kontextu kazde cilove skupiny. 
...
Nejcastejsim problemem v~knihovnich procesech byva lidsky faktor, který zasahuje do systemu. Prikladem mohou byt nespokojeni uzivatele knihoven se sluzbama a personalem knihovny, nebo změna legislativy v~ramci knihovniho zákona ci standardu místo zmeny v~knihovnim systemu. Pro zmenu v~knihovnim systemu ale chybi lide s~patricnym vzdelanim. Dukazem je fakt, ze několik instanci systemu obsluhuje ve skutcnosti jedna osoba která se systemem umi pracovat a také fakt, ze ostatní pracovnici hledaji ruzne zpusoby typu excel, sesit, papir, aby nahradili funkcionalitu Alephu, který neumi po letech obsluhovat. 
Zdalo by se, ze by v~tomto pripade pomohlo skoleni v~ramce práce s~neintuitivnim a ne prácve uzivatelsky privetivym systemem, avsak opak je pravdou, jelikoz i lidi, kteří prosli skolenimi systém stejne neovladaji ani v~kontextu svého vlastniho okruhu práce. 

\section{Popis projektu a jeho etap}

Obsahuje popis hlavních aktivit projektu a jeho etap. Jsou zde zodpovězeny základní
otázky, jaký smysl a zaměření projektu, jaké služby budou díky projektu poskytovány
a jaký problém řeší, kdo je investorem (resp. vlastníkem či provozovatelem) projektu,
jaká je kapacita (velikost) projektu a jaká je jeho lokalizace, jakými etapami projekt
prochází a čím jsou specifické, jak řešeno variantní zpracování v~rámci studie a jaká
jsou ostatní významná specifika projektu. Nezbytné legislativní změny, včetně
harmonogramu přijetí a očekávané účinnosti

Pr pohledu na cilove skupiny, problemy knihoven, zapisky z~vyzkumu a povinnosti, které musí knihovny, hlavne Masarykovy univerzity resit, je potreba vyresit zejména:
napojeni systemu na externi technologie a sluzby 
- Shibboleth
- Ekonomicky system(akvizicni modul)
- personalni systém
- INET
- protokol na vymenu dat (OAI-PMH)
- NCIP jako nahrada za Aleph Restful Api
- selfchecky
discovery sluzby
obalky knihoven (Z39.50)
OPAC (VuFind)
SUPO
IS MU
Centrální portál knihoven

API 
Dokumentaci systemu
praci se systemem
KIC

verejnou cast knihovniho systemu, v~tomto pripade OPAC, který je vlastne vstupni branou uzivatele, která musí byt uzivatelsky privetiva.. a něco, a něco, a něco [Ux Design for Libraries]

\section{Požadavky na knihovní systémem Masarykovy Univerzity}
\textbf{Velký počet knihovnich jednotek, uzivatelu, pobocek podpobocek}\newline

V~teto chvily je potreba , aby nový systém zvladl cca 2 miliony knihovnich jednotek, cca 50 000 aktualne registrovanych  uzivatelu.

Toto kriterium by nemelo byt problemem, jelikoz knihovna v~Turecku a nekde ma x knihovnich jedntek a Y registrvovanych uzivatelu. 
Rozdeleni na pobocky a podpobocky je taky mozne, takto ..\newline\newline

\textbf{Otevrenost systemu}\newline
System musí byt dostatecne otevren na to, aby mohl byt napojen na další systemy vyuzivane v~ramci masarykovy univerzity, tj. Personální, ekonomický, studijní.
Studijni..
Ekonomicky..
personalni..
Celkove fakt, ze Koha je open source systém dava pripadnemu napojovani na další systemy velke moznosti, jelikoz je mozne do kodu zasahovat a prizpusobovat si tak knihovni systém svým potřebám, aniž by knihovna musela žádat výrobce o~novou funkcionalitu.
V~pripade, ze jina knihovna jiz tento krok podstoupila, je mozne v~ramci komunity tento kod  implementovat i do forku te dane knihovny, tudiz není potreba  znovuvynalezat kolo, nebo nesmyslne platit za jiz zaplacenou věc.

\section{Problémy a jejich řešení}
Ux
mobilni OPAC
pristupnost
podpora vufindu, finsko, cpk, open source..

\chapter{Prakticka cast}

\section{Management projektu a projektový tým}

Popisuje způsob řízení projektu z~hlediska lidských zdrojů a projektový tým;,
popis/seznam pracovníků zapojených do projektu, jejich zapojení a pozici v~projektu
(specializaci). Dále zahrnuje veškeré plánování, organizování, řízení a kontrolu všech
procesů a organizačních jednotek, nezbytných pro realizaci aktivit projektu.

\section{Technické a technologické řešení projektu}

Shrnuje veškeré podstatné technické a technologické aspekty projektu, jako je
zvolená technologie, technické parametry jednotlivých zařízení, výhody a nevýhody
těchto předpokládaných řešení, vyplývající technická rizika, potřebné energetické a
materiálové toky, údaje o~životnostech jednotlivých zařízení, potřebné údržbě a
nákladnosti oprav, změny v~provozní náročnosti vlivem opotřebení apod.,
Navrhované metriky (i se zdroji),
Funkční dekompozice
Datová architektura
Procesní architektura
Vymezení služeb poskytovaných systémem
Vymezení zdrojů dat, způsoby iniciálního naplnění systému
Provozní parametry pilotního a ostrého provozu

\section{Způsob zajištění projektu}

Kritéria výběru varianty, jejich popis a zdůvodnění, vyhodnocení variant po
organizační, procesní i technologické stránce, stručný popis nejvhodnější varianty,
stručné zdůvodnění výběru varianty

\section{ Zajištění investičního (dlouhodobého) majetku}

vymezení struktury dlouhodobého majetku, určení výše investičních nákladů,
problematika servisních podmínek a případného znovupořízení, amortizační schéma
apod.
Doložení úpravy vlastnických vztahů dle řídící dokumentace IOP (požadavků
stanovených výzvou a Příručkou pro žadatele a příjemce).

\section{Harmonogram realizace projektu včetně rozpočtového harmonogramu}

časový plán jednotlivých činností a fází projektu, který by měl být zpracován do
podoby harmonogramu. Mělo by z~něj být patrné, kde jednotlivé činnosti začínají a
kdy končí (pokud končí), které činnosti na které navazují a jaké se vzájemně
překrývají.

\section{Finanční a ekonomická analýza }

finanční plán investiční etapy
- plán průběhu provozních, investičních nákladů a výnosů
finanční plán provozní etapy
- plán průběhu provozních, investičních nákladů a výnosů
plánované stavy majetku
plán průběhu cash-flow – výdajů a příjmů
vyhodnocení finanční analýzy
Ekonomická analýza
cost benefit analýza (CBA)
- vymezení všech zainteresovaných subjektu a jejich členění
- popis investiční a nulové varianty
- popis ocenitelných nákladu a přínosů
- popis nákladů a přínosů nezahrnovaných do CBA
- výpočet hodnoty přínosů a nákladů
- výpočet kriteriálních ukazatelů
- provedení citlivostní analýzy
- celkový ekonomický peněžní tok
vyhodnocení ekonomické analýzy
- interpretace výsledku a rozhodnutí o~přijatelnosti investice,
financovatelnosti a udržitelnosti

\section{Hodnocení efektivity a udržitelnosti projektu}

vyhodnocení projektu pomocí kriteriálních ukazatelů kalkulovaných z~finančních toků
(resp. nákladů, výnosů) jako např. NPV, IRR, Doba návratnosti, Index rentability a
finanční analýza projektu. U~projektů, které negenerují příjmy a nelze u~nich vypočítat
ukazatele finanční analýzy, musí být podrobně zdůvodněno, kdo bude zabezpečovat 
provoz investice a z~jakých zdrojů budou kryty provozní náklady po ukončení
realizace projektu.

\section{Analýza a řízení rizik}

identifikace rizik - vymezení největších zdrojů rizika v~projektu (hlavních rizik
v~oblasti organizační, procesní, technologické, implementační, informační atd.),
popis možných následků při realizaci rizika
- odhad pravděpodobnosti realizace rizik na základe historických dat nebo ze
simulačních modelů, ohodnocení rizik na základě jejich následků a pravděpodobnosti
jejich realizace, návrh opatření na jejich snížení nebo eliminaci – organizační,
procesní, technologické a další opatření
- náklady spojené s~těmito opatřeními

\section{ Vliv projektu na životní prostředí vliv a vliv projektu na rovné příležitost}

Koha ma mene Kodu nez Aleph, setri zivotni prostredi.

\chapter{Zhodnocení projektu na základě výsledků studie.}

 zahrnuje popis zásadních závěrů, které vyplývají ze zpracované studie
proveditelnosti. V~tabulce uveďte zásadní ukazatele a jejich hodnoty spočtené
z~výsledných hotovostních toků resp. nákladů a výnosů obsažených ve finálním
finančním plánu, jakož i výsledky citlivostní analýzy. Ve stručné a shrnující podobě je
zde uvedeno zhodnocení finanční efektivity projektu, jeho realizovatelnost z~hlediska
všech prvků Studie proveditelnosti a výsledky analýzy rizik. 

\chapter{Zaver}

%%%%%%%%%%%%%%%%%%%%%%%%%%%%% TABLE OF CONTENTS %%%%%%%%%%%%%%%%%%%%%%%%%%%%%

\makeatletter\thesis@blocks@clear\makeatother
\phantomsection %% Print the index and insert it into the
\addcontentsline{toc}{chapter}{\indexname} %% table of contents.

%%%%%%%%%%%%%%%%%%%%%%%%%%%% APPENDIX HYPERREFS %%%%%%%%%%%%%%%%%%%%%%%%%%%%

\makeatletter\thesis@blocks@clear\makeatother
%% Patch the appendix hyperrefs (see:
%% <http://tex.stackexchange.com/q/174887/70941>)
\renewcommand{\theHchapter}{A\arabic{chapter}}
\appendix %% and start the appendices.

%%%%%%%%%%%%%%%%%%%%%%%%%%%%%%% BIBLIOGRAPHY %%%%%%%%%%%%%%%%%%%%%%%%%%%%%%%

\printbibliography[title={Seznam literatury},heading=bibintoc]

%%%%%%%%%%%%%%%%%%%%%%%%%%%%%% END OP CONTENT %%%%%%%%%%%%%%%%%%%%%%%%%%%%%%

}
\end{document}
