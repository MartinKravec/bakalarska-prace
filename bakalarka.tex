%%%%%%%%%%%%%%%%%%%%%%%%%%%%%%%%%%%%%%%%%%%%%%%%%%%%%%%%%%%%%%%%%%%%
%% Author: Martin Kravec <428724@mail.muni.cz>
%% UČO: 428724
%%%%%%%%%%%%%%%%%%%%%%%%%%%%%%%%%%%%%%%%%%%%%%%%%%%%%%%%%%%%%%%%%%%%

% @TODO: Odsazeni

\documentclass[
	12pt, oneside, printed, final, 
	table,   %% Causes the coloring of tables. Replace with `notable` to restore plain tables.
	lof,     %% Prints the List of Figures. Replace with `nolof` to hide the List of Figures.
	lot     %% Prints the List of Tables. Replace with `nolot` to hide the List of Tables.
	%% More options are listed in the user guide at
	%% <http://mirrors.ctan.org/macros/latex/contrib/fithesis/guide/mu/phil.pdf>.
]{fithesis3}

%%%%%%%%%%%%%%%%%%%%%%%%%%% DEPENDENCIES %%%%%%%%%%%%%%%%%%%%%%%%%%%

\usepackage[czech]{babel}
\usepackage[T1]{fontenc}
\usepackage[utf8]{inputenc}
\usepackage[plainpages=false,pdfpagelabels,unicode]{hyperref}
\usepackage{bold-extra}
\usepackage{setspace}
\usepackage{lastpage}
\usepackage{anyfontsize}
\usepackage{graphicx}

%%%%%%%%%%%%%%%%%%%%%%%%%%% THESIS SETUP %%%%%%%%%%%%%%%%%%%%%%%%%%%

\thesissetup{
	university    = mu,
    faculty       = phil,
    department    = Kabinet informačních studií a knihovnictví,
    field         = Informační studie a knihovnictví,
    type          = bc,
    author        = Martin Kravec,
    gender        = m,
    advisor       = Mgr. Michal Denár,
    date			  = 2016/03/12,
    keywords	  	  = {open source, Koha, Aleph, knihovní system},
	TeXkeywords	  = {open source, Koha, Aleph, library system},
	%keywords	En    = {open source, Koha, Aleph, library system},
	TeXkeywordsEn = {open source, Koha, Aleph, library system},
    title         = Studie proveditelnosti implementace open source integrovaného knihovního systému Koha v~knihovnách Masarykovy univerzity,
    titleEn		  = Feasibility study for the implementation of open source integrated library system Koha in libraries of Masaryk University,
    TeXtitle      = Studie proveditelnosti implementace open source integrovaného knihovního systému Koha v~knihovnách Masarykovy univerzity,
    TeXtitleEn      = Studie proveditelnosti implementace open source integrovaného knihovního systému Koha v~knihovnách Masarykovy univerzity
}

%%%%%%%%%%%%%%%%%%%%%%%%%%%%%% THANKS %%%%%%%%%%%%%%%%%%%%%%%%%%%%%%

\thesislong{thanks}{
    Chtěl bych poděkovat vedoucímu práce Mgr. Michalovi Denárovi za to, že mi ukázal, jak krásně by mohlo fungovat české knihovnictví, a také Mgr. Petře Žabičkové v~roli konzultantky za to, že mi ukázala, jak je na tom české knihovnictví ve skutečnosti. Dále bych rád poděkoval RNDr. Michalovi Růžičkovi, který se chopil role konzultanta a poskytl nedocenitelné informace. A~v~neposlední řadě si zaslouží můj vděk také zaměstnanci univerzitních knihoven MU, kteří jako respondenti ochotně spolupracovali při provádění rozhovorů a stínování v~rámci mého průzkumu mapování knihovních procesů v~knihovnách Masarykovy univerzity.
}

\thesislong{abstract}{ % ANNOTATION!
    Cílem této práce je zejména provést studii proveditelnosti, která určí, zda je možná migrace z~knihovního systému Aleph na open source integrovaný knihovní systém Koha v~knihovnách Masarykovy univerzity. Celá studie je založena na šestiměsíčním průzkumu týkajícího se mapování knihovních procesů ve vybraných knihovnách univerzity, a to formou rozhovorů a stínování. Průzkum byl také prováděn i v~síti 900 knihoven v~Turecku. Studie v~úvodu rozebírá analýzu trhu, na kterou pak navazuje finanční analýza propojená s~řízením a minimalizací rizik. Součástí studie proveditelnosti je také evaluace efektivity a udržitelnosti projektu založena na SWOT analýze. Práce také pojednává o~projektovém managementu a řízení lidských zdrojů a podrobněji rozebírá technické a technologické řešení projektu, přičemž objasňuje aspekty vývoje open source softwaru. 
}

\thesislong{abstractEn}{ % EN ANNOTATION!
    Cílem této práce je zejména provést studii proveditelnosti, která určí, zda je možná migrace z~knihovního systému Aleph na open source integrovaný knihovní systém Koha v~knihovnách Masarykovy univerzity. Celá studie je založena na šestiměsíčním průzkumu týkajícího se mapování knihovních procesů ve vybraných knihovnách univerzity, a to formou rozhovorů a stínování. Průzkum byl také prováděn i v~síti 900 knihoven v~Turecku. Studie v~úvodu rozebírá analýzu trhu, na kterou pak navazuje finanční analýza propojená s~řízením a minimalizací rizik. Součástí studie proveditelnosti je také evaluace efektivity a udržitelnosti projektu založena na SWOT analýze. Práce také pojednává o~projektovém managementu a řízení lidských zdrojů a podrobněji rozebírá technické a technologické řešení projektu, přičemž objasňuje aspekty vývoje open source softwaru. 
}

%%%%%%%%%%%%%%%%%%%%%%%%%%% BIBLIOGRAPHY %%%%%%%%%%%%%%%%%%%%%%%%%%%

\usepackage{csquotes}
\usepackage[
	backend 		= biber,
  	sortlocale	= cs_CZ,
  	style		= iso-authortitle,
  	bibencoding = UTF8,
  	babel		= other	% to support multiple languages in bibliography
]{biblatex}
\addbibresource{bibliography.bib}
%% `style`s and `citestyles`, see:
%% <http://mirrors.ctan.org/macros/latex/contrib/biblatex/doc/biblatex.pdf>.

%%%%%%%%%%%%%%%%%%%%%%%%%%% NEW CITATION COMMANDS %%%%%%%%%%%%%%%%%%%%%%%%%%%

\newcommand{\citepages}[2]{[\cite[#1]{#2}]}

\newcommand{\citesource}[1]{[\cite{#1}]}


%%%%%%%%%%%%%%%%%%%%%%%%%%% NEW FORMATING COMMANDS %%%%%%%%%%%%%%%%%%%%%%%%%%%

\newcommand{\bold}[1]{\textbf{#1}}
\newcommand{\italic}[1]{\textit{#1}}
%\newcommand{\citation}[1]{„\italic{#1}"} % @TODO proc nefunguje?

%%%%%%%%%%%%%%%%%%%%%%%%%%%%%% INDEX %%%%%%%%%%%%%%%%%%%%%%%%%%%%%%

\makeindex

%%%%%%%%%%%%%%%%%%%%%%%%%%%% PROHLASENI %%%%%%%%%%%%%%%%%%%%%%%%%%%

\makeatletter\thesis@load
  \makeatletter\def\thesis@czech@declaration{%
  Prohlašuji, že jsem předkládanou práci zpracoval%
  \thesis@czech@gender@koncovka\ samostatně~a použil%
  \thesis@czech@gender@koncovka\ jen uvedené prameny~a
  literaturu. Současně dávám svolení k~tomu, aby
  elektronická verze této práce byla zpřístupněna přes
  informační systém Masarykovy univerzity.}
\makeatother

%%%%%%%%%%%%%%%%%%%%% DOCUMENT STRUCTURE %%%%%%%%%%%%%%%%%%%%

\makeatletter
  \def\thesis@blocks@preamble{%
    \thesis@blocks@coverMatter
  \thesis@blocks@cover
  \thesis@blocks@titlePage
    \thesis@blocks@frontMatter
      {\Large{Bibliografický záznam}}\newline\newline
        KRAVEC, Martin. \italic{\thesis@title}. Brno: Masarykova univerzita, Filosofická fakulta, 2016. \pageref{LastPage} s. Vedoucí diplomové práce Mgr. Michal Denár.
	  \newline\newline
      {\Large{Anotace}}\newline\newline
        \thesis@abstract\newline\newline
      {\Large{Klíčová slova}}\newline\newline
        \thesis@keywords\newline\newline
      {\Large{Annotation}}\newline\newline
        \thesis@abstractEn\newline\newline
      {\Large{Keywords}}\newline\newline
        \thesis@TeXkeywordsEn\newline\newline
  \thesis@blocks@declaration
  \thesis@blocks@thanks
  \thesis@blocks@clear
	\tableofcontents}
  \def\thesis@blocks@postamble{%
    \thesis@blocks@lot
    \thesis@blocks@lof}
\makeatother

% Je potreba dotahnout tikz.
%\thesis@require{tikz}
%\thesis@require{geometry}
%\geometry{top=15mm,bottom=10mm,left=15mm,right=15mm,includeheadfoot}
  
%%%%%%%%%%%%%%%%%%%%%%%%%%%%% DOCUMENT %%%%%%%%%%%%%%%%%%%%%%%%%%%%%
\begin{document}{\fontsize{12}{18}

\chapter{Uvod}

Tato bakalářská diplomová práce se bude zabývat studií proveditelnosti implementace open source integrovaného knihovního systému Koha v~knihovnách Masarykovy univerzity. V~práci bude probíráno několik úhlů pohledu na celkovou problematiku, a to v~kontextu studie proveditelnosti.  V~práci tedy rozebereme nejdřív teoretickou část knihovních systémů, co by měly umět a jaký je jejich cíl jako produktu softwarových firem, to znamená, že probereme technické zastřešení projektu a vysvětlíme veškeré pojmy s~tím spojené. Teoretická část dále vysvětlí oblast projektového managmentu a přiblíží problematiku migrací na nový knihovní systém. Celý projekt bude řešen jako harmonogram realizace spolu s~finanční a ekonomickou analýzou. V~rámci studie byl prováděn několikaměsíční průzkum v~knihovnách Masarykovy univerzity a průzkum v~síti 900 knihoven v~Turecku. Při průzkumu byla realizována také SWOT analýza, na které je pak založeno celé hodnocení efektivity a udržitelnosti projektu, ve kterém nebude chybět analýza a řízení případných rizik, které s~sebou projekt migrace nese. V~plánu je také zkrácený výstup v~angličtině ve formě článku, jelikož je to téma s~globálním dopadem, které může pomoci i institucím v~zahraničí, které řeší stejný nebo podobný problém.

\chapter{Teoretická část}

\section{Projektové řízení jako takové}

\subsection{Projekt}

Projektové řízení neboli projektový management se zabývá řízením projektů. Jeho cíle lze rozdělit na 3 části: 
\bigskip
\begin{itemize}
\item splnění požadavků
\item časový plán
\item rozpočtové náklady
\end{itemize}

Tomuto rozdělení se také říká trojimperativ\citepages{5}{rosenau_2000}. O~každém projektu lze prohlásit, že je jedinečný, jelikož je prováděn jen jednou, pracují na něm jiní lidé a je časově ohraničen. Projekt je realizován pomocí lidských a materiálních zdrojů. To znamená, že projektový manager vlastně řídí lidi tak, aby byly efektivně využity materiální zdroje\citepages{28}{rehacek_2013}. Následující obrázek 1.1 znázorňuje kvalitu provedení v~kontextu trojimperativu.% @TODO label na obrazek

\begin{figure}[H]
\centering
%egraphics[width=0.95\textwidth,frame]{Resources/Trojimperativ} % @TODO obrazek
\caption{Důsledky trojimperativu.}
\end{figure}

Projekt může být charakterizován jako hmotný či nehmotný projekt v~závislosti na konečném produktu. V~případě hmotného projektu by šlo například o~hardware, u~nehmotného o~software. Produkt je tedy konečným výstupem každého projektu.
V~případě, kdy je zákazník mimo organizaci, která projekt realizuje, jsou projekty realizovány na základě uzavřených smluv, výsledný produkt je tedy posuzován na základě požadavků zákazníka. Je-li zákazníkem stejná organizace, není potřeba uzavírat smlouvy. V~tom případě je ale výstup projektu posuzován podle času dokončení a návratnosti investice\citepages{10-12}{rosenau_2000}.  V~projektovém řízení se zákazník jako objednatel projektu označuje jako zadavatel projektu. Způsob provedení projektu také závisí na konkurenci. Situace, kdy je produkt na trhu jediný svého druhu, je úplně jiná, než u~produktu, kterých je v~oboru mnoho.\citepages{12}{rosenau_2000}
Projektovému řízení se také říká “řízení projektů” nebo “řízení programu” v~závislosti na velikosti projektu. Obecně platí, že programy jsou větší než projekty a projekty větší než úkoly.\citepages{12}{rosenau_2000}
Úspěšnost projektu je závislá na kompetencích managera (nebo leadera) projektu\citepages{1}{Muller2010437}%@TODO pridat strany
, které lze rozdělit na: 
\bigskip
\begin{enumerate}
\item Managerské kompetence
\begin{itemize}
\item Efektivní řízení zdrojů
\item Podporování komunikace v~týmu
\item Zmocnění (leader dává svým přímým podřízeným autonomii při řešení problémů, čímž přispívá k~rozvoji jejich \item vlastní zodpovědnosti)
\item Rozvoj (leader nabádá ostatní, aby si brali stále náročnější úkoly a sám investuje svůj čas a úsilí do rozvoje jejich kompetencí)
\item Odhodlání (leader ukazuje své neochvějné odhodlání při dosahování cílů a realizaci projektu)
\end{itemize}
\item Intelektuální kompetence
\begin{itemize}
\item Kritické myšlení (sběr relevantních informací z~různých zdrojů a hledání výhodných a nevýhodných řešení)
\item Vize a představivost (leader má jasnou vizi o~budoucnosti projektu)
\item Strategická perspektiva (leader si je vědom problémů a jejich důsledků, nachází příležitosti a hrozby)
\end{itemize}
\item Emocionální kompetence
\begin{itemize}
\item Sebepoznání (leader je schopen ovládat své pocity)
\item Emocionální odolnost (leader je schopen udržovat stálý výkon ve všech situacích)
\item Intuitivnost
\item Motivace (leader musí svými činy působit motivačně)
\item Svědomitost (leader ukazuje svoji angažovanost k~projektu a sám povzbuzuje ostatní členy pracovní skupiny)
\end{itemize}
\end{enumerate}

\subsection{Proces řízení projektu}
Řízení projektu sestává z~procesu pěti managerských činností, a to\citepages{22-23}{rehacek_2013}:
\bigskip
\begin{itemize}
\item Definování projektových cílů
\item Plánování splnění cílů, tedy trojimperativu
\item Efektivní vedení lidských zdrojů (podřízení, dodavatelé)
\item Monitorování - sledování odchylek od plánu či stavů jednotlivých fází projektu
\item Ukončení - finalizace ve smyslu kontroly, zda produkt odpovídá definicím zadavatele, dokončení prací (například dokumentace)
\end{itemize}

Vzájemné závislosti v~procesu řízení projektu znázorňuje obrázek 1.2. % @TODO label na obrazek
Celému cyklu však předchází tzv. předprojektová fáze, do které patří zejména studie proveditelnosti, o~které pojednává další kapitola.

\begin{figure}[H]
\centering
%egraphics[width=0.95\textwidth,frame]{Resources/project-management} % @TODO obrazek
\caption{Vzájemné závislosti v~procesu řízení projektu.}
\end{figure}

\section{Studie proveditelnosti obecně}

Hlavním účelem studie proveditelnosti (feasibility study) je zhodnotit možné varianty, které mohou nastat při provádění projektu, a posoudit realizovatelnost a následnou životaschopnost zvoleného řešení. Studie proveditelnosti zpřesňuje vlastnosti projektu a to hlavně specifikaci cíle, potřebné finanční, materiální a lidské zdroje, časový harmonogram, přínosy a rizika spojená s~realizací projektu. Jelikož studie pojednává o~finančních, technických, managerských i ekonomických aspektech projektu, nazývá se také studií ekonomicko-technickou\citepages{19}{fotr_1995}.
Vyhodnocení studie tedy vyúsťuje do rozhodnutí o~zamítnutí či přijetí projektu a případně jeho následné realizaci.
\citepages{19-20}{fotr_1995} 

\bigskip

„\italic{Významné je, aby studie co nejlépe popisovala, variantně řešila, optimalizovala a hodnotila investiční projekt se všemi z~něj vyplývajícími specifiky.}“\citepages{8}{Sieber2004} 

\bigskip

Celková osnova studie proveditelnosti by měla vypadat přibližně takto:\citepages{8-14}{Sieber2004} 
\bigskip
\begin{enumerate}
\item \bold{Obsah} - struktura kapitol
\item \bold{Úvodní informace} - účel zpracování studie proveditelnosti
\item \bold{Stručné vyhodnocení projektu} - závěry studie v~rozsahu 1-2 stran, zhodnocení finanční efektivity projektu
\item \bold{Stručný popis podstaty projektu a jeho etap} - komplexně pojednává o~hlavních rysech projektu (název, smysl, zaměření projektu, výsledný produkt a problémy, které produkt řeší, lokalizace a etapy projektu)
\item \bold{Analýzy trhu, odhad poptávky, marketingová strategie a marketingový mix} - marketingové aspekty (potřeby finálních uživatelů produktu, konkurenceschopnost produktu)
\item \bold{Management projektu a řízení lidských zdrojů} - plánování, organizace a management procesů a lidských zdrojů projektu 
\item \bold{Technické a technologické řešení projektu} - výhody a nevýhody zvolených technologií, materiálové a energetické toky, technická rizika, náklady na údržbu, správu a provoz.
\item \bold{Dopad projektu na životní prostředí} - kladné i negativní vlivy jednotlivých etap realizace projektu
\item \bold{Zajištění dlouhodobého majetku} - výše investičních nákladů, struktura dlouhodobého majetku
\item \bold{Řízení pracovního kapitálu (oběžný majetek)} - velikost a struktura oběžného majetku
\item \bold{Finanční plán a analýza projektu} - celkové zobecnění předchozích bodů
\item \bold{Hodnocení efektivity a udržitelnosti projektu} - evaluace projektu na základě zadaných kritérií, finanční toky a doba návratnost investic
\item \bold{Řízení rizik (citlivostní analýza)} - výčet zdrojů rizik projektu, opatření
\item \bold{Harmonogram projektu} - časový harmonogram jednotlivých etap projektu. Začátky a konce jednotlivých činností
\item \bold{Podrobné závěrečné hodnocení projektu} - komplexní vyjádření k~realizovatelnosti projektu
\end{enumerate}

Vyhodnocení finanční rentability projektu vybranými ukazateli se provádí tak, že se nejdřív určí jednorázové náklady na investice, odhadnou se budoucí výnosy z~investice, následně se určí náklady na kapitál a nakonec se vypočítá současná hodnota očekávaných výnosů. Na to se pak různě aplikují metody ekonomického hodnocení investice.\citepages{17-18}{Podesvova2010thesis}

\section{Open source jako systém}
Open source systém lze chápat ve dvou rovinách, a to z~pohledu teorie systémů jako systém, který je založen na otevřeném svobodném kódu a slouží pro využívání, zpracování a zprostředkování informací, a také z~pohledu systematičnosti vývoje, kam patří správa verzí, bug trackery, financování, licencování, vytváření balíčků či spolupráce s~komunitou.\citepages{40}{cejpek_2005}

\subsection{Open source software}

Pojem „open source software” bývá častokrát nesprávně zaměňován s~pojmem „free software”, je proto důležité vymezit si rozdíly v~těchto pojmech. 

\bold{Open source software} (OSS) je software, jehož zdrojový kód může být kýmkoliv upravován nebo vylepšován. Open source licence totiž povoluje legální přístup, možnost modifikace či další distribuce tohoto softwaru. Nemalá část komerčního softwaru vychází právě z~open source projektů.

\bold{Free software} je software poskytovaný zdarma, což však neznamená, že má uživatel licenci k~přístupu, modifikaci či k~další distribuci tohoto softwaru. Je definován spíše v~kontextu svobody než ceny. 

Opakem free softwaru je software, jehož zdrojový kód může být modifikován pouze jeho autorem či organizací, která ho vydala. Tomu se říká \bold{closed source software} neboli \bold{proprietární software} (od slova „property”, což znamená, že zdrojový kód softwaru je majetkem svých autorů a jenom oni mohou legálně kopírovat či modifikovat tento software). Příkladem takového softwaru je například Microsoft Office.% @TODO add citation
% McLean, A. (2015). Open-source software. Canadian Journal of Nursing Informatics, 10(3) Retrieved from http://ezproxy.techlib.cz/login?url=http://search.proquest.com/docview/1753599175?accountid=119841

Všeobecně lze OSS licence rozdělit do tří kategorií a to:\citepages{9-10}{6226510}
\begin{itemize}
\item restriktivní neboli \bold{copyleft} licence (omezující)
\item středně restriktivní neboli \bold{copycenter} licence
\item \bold{permisivní} licence (tolerantní, shovívavé)
\end{itemize}
Teoreticky však existuje nekonečné množství těchto licencí, jelikož si autoři open source softwaru mohou vybrat jakoukoliv licenci nebo si vytvořit licenci vlastní.
Na základě výše řečeného lze tvrdit, že open source software je podmnožinou free softwaru, avšak free software nemusí být open source.

%%%%%%%%%%%%%%%%%%%%%%%%%%%%% TABLE OF CONTENTS %%%%%%%%%%%%%%%%%%%%%%%%%%%%%

\makeatletter\thesis@blocks@clear\makeatother
\phantomsection %% Print the index and insert it into the
\addcontentsline{toc}{chapter}{\indexname} %% table of contents.

%%%%%%%%%%%%%%%%%%%%%%%%%%%% APPENDIX HYPERREFS %%%%%%%%%%%%%%%%%%%%%%%%%%%%

\makeatletter\thesis@blocks@clear\makeatother
%% Patch the appendix hyperrefs (see:
%% <http://tex.stackexchange.com/q/174887/70941>)
\renewcommand{\theHchapter}{A\arabic{chapter}}
\appendix %% and start the appendices.

%%%%%%%%%%%%%%%%%%%%%%%%%%%%%%% BIBLIOGRAPHY %%%%%%%%%%%%%%%%%%%%%%%%%%%%%%%

\printbibliography[title={Seznam literatury},heading=bibintoc]

%%%%%%%%%%%%%%%%%%%%%%%%%%%%%% END OP CONTENT %%%%%%%%%%%%%%%%%%%%%%%%%%%%%%

}
\end{document}
