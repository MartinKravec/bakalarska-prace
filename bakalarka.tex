%%%%%%%%%%%%%%%%%%%%%%%%%%%%%%%%%%%%%%%%%%%%%%%%%%%%%%%%%%%%%%%%%%%%
%% Author: Martin Kravec <428724@mail.muni.cz>
%% UČO: 428724
%%%%%%%%%%%%%%%%%%%%%%%%%%%%%%%%%%%%%%%%%%%%%%%%%%%%%%%%%%%%%%%%%%%%

\documentclass[
	12pt, oneside, printed, final, 
	table,   %% Causes the coloring of tables. Replace with `notable` to restore plain tables.
	lof,     %% Prints the List of Figures. Replace with `nolof` to hide the List of Figures.
	lot     %% Prints the List of Tables. Replace with `nolot` to hide the List of Tables.
	%% More options are listed in the user guide at
	%% <http://mirrors.ctan.org/macros/latex/contrib/fithesis/guide/mu/phil.pdf>.
]{fithesis3}

%%%%%%%%%%%%%%%%%%%%%%%%%%% DEPENDENCIES %%%%%%%%%%%%%%%%%%%%%%%%%%%

\usepackage[czech]{babel}
\usepackage[T1]{fontenc}
\usepackage[utf8]{inputenc}
\usepackage[plainpages=false,pdfpagelabels,unicode]{hyperref}

%%%%%%%%%%%%%%%%%%%%%%%%%%% THESIS SETUP %%%%%%%%%%%%%%%%%%%%%%%%%%%

\thesissetup{
	university    = mu,
    faculty       = phil,
    department    = Kabinet informačních studií a knihovnictví,
    field         = Informační studie a knihovnictví,
    type          = bc,
    author        = Martin Kravec,
    gender        = m,
    advisor       = Mgr. Michal Denár,
    date			  = 2016/03/12,
	TeXkeywords={open source, Koha, Aleph, library system},
    title         = Studie proveditelnosti implementace open source integrovaného knihovního systému Koha v knihovnách Masarykovy univerzity,
    TeXtitle      = Studie proveditelnosti implementace open source integrovaného knihovního systému Koha v knihovnách Masarykovy univerzity,
    titleEn		  = Feasibility study for the implementation of open source integrated library system Koha in libraries of Masaryk University
}

%%%%%%%%%%%%%%%%%%%%%%%%%%%%%% THANKS %%%%%%%%%%%%%%%%%%%%%%%%%%%%%%

\thesislong{thanks}{
    Děkuji
}

%%%%%%%%%%%%%%%%%%%%%%%%%%% BIBLIOGRAPHY %%%%%%%%%%%%%%%%%%%%%%%%%%%

\usepackage{csquotes}
\usepackage[
	backend 		= biber,
  	sortlocale	= cs_CZ,
  	style		= numeric,
  	citestyle	= numeric-comp
]{biblatex}
\addbibresource{bibliography.bib}
%% `style`s and `citestyles`, see:
%% <http://mirrors.ctan.org/macros/latex/contrib/biblatex/doc/biblatex.pdf>.

%%%%%%%%%%%%%%%%%%%%%%%%%%% INDEX %%%%%%%%%%%%%%%%%%%%%%%%%%%

\makeindex

%%%%%%%%%%%%%%%%%%%%%%%%%%%%% DOCUMENT %%%%%%%%%%%%%%%%%%%%%%%%%%%%%

\begin{document}

\chapter{Introduction}

Bakalářská diplomová práce se bude zabývat studií proveditelnosti implementace 
open source integrovaného knihovního systému Koha v knihovnách Masarykovy univerzity. 
Aktuálnost problematiky tkví v tom, že společnost Ex Libris, která vyvíjí knihovní 
systém Aleph, přestává tomuto produktu poskytovat podporu a zároveň vytváří produkt jiný, 
jenž však funguje výhradně v cloudu. 
Mimo fakt, že data již nebudou ukládána lokálně, to znamená, že knihovny budou mít 
menší možnosti úprav a rozšíření své instance systému. Tyto problémy by pak měla 
vyřešit právě migrace na knihovní systém Koha. 
Další důvod podporující migraci je fakt, že knihovny platí milionové částky za 
podporu stávajícího systému, což je oproti open source systémům markantní rozdíl, 
který lze investovat do smysluplnějších  věcí spojených s cíli knihoven, 
jakožto vzdělávacími, informačními a kulturními institucemi. 

\chapter{Teoreticka cast }

\section{zakladni pilire knihovniho systemu}

Jaky by mel byt, co by mel splnovat

\section{Proc prave Koha}

protoze.

\chapter{Prakticka cast}

\section{Vyzkum}

probehl \cite{L36Hsi0FBbF4yvCA}

\section{Studie proveditelnosti}

proveditelne

\chapter{Zaver}

zaver

%%%%%%%%%%%%%%%%%%%%%%%%%%%%% TABLE OF CONTENTS %%%%%%%%%%%%%%%%%%%%%%%%%%%%%

\makeatletter\thesis@blocks@clear\makeatother
\phantomsection %% Print the index and insert it into the
\addcontentsline{toc}{chapter}{\indexname} %% table of contents.

%%%%%%%%%%%%%%%%%%%%%%%%%%%% APPENDIX HYPERREFS %%%%%%%%%%%%%%%%%%%%%%%%%%%%

\makeatletter\thesis@blocks@clear\makeatother
%% Patch the appendix hyperrefs (see:
%% <http://tex.stackexchange.com/q/174887/70941>)
\renewcommand{\theHchapter}{A\arabic{chapter}}
\appendix %% and start the appendices.

%%%%%%%%%%%%%%%%%%%%%%%%%%%%%%% BIBLIOGRAPHY %%%%%%%%%%%%%%%%%%%%%%%%%%%%%%%

\printbibliography[title={Seznam literatury},heading=bibintoc]

\end{document}
